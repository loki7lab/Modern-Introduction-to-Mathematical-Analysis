\ifx\allfiles\undefined% 如果这个没定义主文件
    \documentclass[12pt,a4paper,oneside,utf8]{ctexbook}
    \def\basepath{../}%basepath =  '../config'
    % -------------------------- 1. 路径与外部配置 --------------------------
% 定义基础路径变量(未提前定义则默认当前目录)
\ifx\basepath\undefined
    \newcommand{\basepath}{./}
\fi
% 引入外部宏包配置文件(按基础路径拼接)
% config/package.tex 
% 列出模板加载的包清单


% 基础数学和工具
\usepackage{amsmath} 
\usepackage{amsthm} 
\usepackage{amssymb}
\usepackage{mathrsfs} 
\usepackage{esint} 
\usepackage{yhmath} 
\usepackage{extarrows}
\usepackage{indentfirst}

% 颜色 (XColor 必须在 hyperref 之前)
\usepackage[dvipsnames, svgnames]{xcolor}
\usepackage[svgnames]{xcolor}
\usepackage{ulem} % 替代 soul



% 字体设置(fontspec 应该在 CJK 字体设置之前)
\usepackage{fontspec} 
\setCJKmainfont{Songti SC} 
\setmonofont{Courier New}
\setmainfont{Times New Roman}


% 布局和样式
\usepackage{geometry}
\usepackage{graphicx} 
\usepackage{fancyhdr} 
\usepackage{enumitem}
\usepackage[strict]{changepage} 
\usepackage{framed} 
\usepackage{listings}
% listings 依赖 xcolor,顺序没问题

\usepackage{setspace}  
\usepackage{microtype} 
\usepackage{titlesec}  

% 目录定制 (tocloft 必须在 hyperref 之前)
\usepackage{tocloft} 

% 超链接 (hyperref 必须是最后一个主要的宏包)
\usepackage{hyperref}

% 设置图片文件根路径(基础路径下的figure文件夹)
\graphicspath{{\basepath figure/}}


% -------------------------- 2. 页面尺寸与边距 --------------------------
\geometry{ 
    papersize={210mm,285mm}, % 页面尺寸(接近A4,略短)
    left=22mm, right=22mm,   % 左右边距
    top=28mm, bottom=28mm,   % 上下边距
    headheight=14pt, headsep=20pt % 页眉高度与间距
}


% -------------------------- 3. 章节标题样式 --------------------------
% 部分(Part)标题:新页+居中+特大号加粗
\titleformat{\part}{\newpage\centering\Huge\bfseries}{}{0em}{}
\titlespacing*{\part}{0pt}{0ex}{3.0ex} % 部分标题间距

% 章节(Chapter)标题:居中+大号加粗+带章节号
\titleformat{\chapter}{\centering\Large\bfseries}{\thechapter.}{1em}{}
\titlespacing*{\chapter}{0pt}{3.0ex}{2.0ex} % 章节标题间距

% 小节(Section)标题:大号加粗+带§符号
\titleformat{\section}{\centering\large\bfseries}{§\arabic{section}}{0.8em}{}
\titlespacing*{\section}{0pt}{2.5ex}{1.5ex} % 小节标题间距

% 自定义未编号章节(Explanation):封装chapter*,复用其样式
\newcommand{\explanation}[1]{\chapter*{#1}}



% -------------------------- 4. 段落与列表格式 --------------------------
\parindent=2em % 段落首行缩进2字符

% 有序/无序/描述列表统一格式:调整间距
\setenumerate[1]{itemsep=5pt, partopsep=0pt, parsep=\parskip, topsep=5pt}
\setitemize[1]{itemsep=5pt, partopsep=0pt, parsep=\parskip, topsep=5pt}
\setdescription{itemsep=5pt, partopsep=0pt, parsep=\parskip, topsep=5pt}

% 重新定义 \hl 命令
\definecolor{lightpink}{rgb}{1.0, 0.71, 0.76}
\newcommand{\hl}[1]{%
  \bgroup
  \markoverwith{\textcolor{lightpink}{\rule[-.5ex]{2pt}{2.5ex}}}%
  \ULon{#1}%
}


% -------------------------- 5. 页眉页脚配置 --------------------------
% -------------------------- 5. 页眉页脚配置 --------------------------
\setlength{\headheight}{15pt} % 解决 headheight too small 警告
\pagestyle{fancy} % 使用fancy样式
\fancyhf{} % 清空默认内容
\fancyhead[L]{\leftmark} % 左页眉:当前章节标题
\fancyhead[R]{\thepage} % 右页眉:页码
\renewcommand{\headrulewidth}{0.4pt} % 页眉下边框线宽
\renewcommand{\footrulewidth}{0pt} % 隐藏页脚下边框


% -------------------------- 6. 代码块样式(分语言定义) --------------------------
% 定义Mathematica代码样式(避免全局覆盖)
\lstdefinestyle{MathematicaStyle}{
    language=Mathematica, basicstyle=\tt, breaklines=true,
    keywordstyle=\bfseries\color{NavyBlue}, emphstyle=\bfseries\color{Rhodamine},
    commentstyle=\itshape\color{black!50!white}, stringstyle=\bfseries\color{PineGreen!90!black},
    columns=flexible, numbers=left, numberstyle=\footnotesize, frame=tb, breakatwhitespace=false
}

% 定义TeX代码样式(避免全局覆盖)
\lstdefinestyle{TeXStyle}{
    language=TeX, basicstyle=\ttfamily, breaklines=true,
    keywordstyle=\bfseries\color{NavyBlue}, emphstyle=\bfseries\color{Rhodamine},
    commentstyle=\itshape\color{black!50!white}, stringstyle=\bfseries\color{PineGreen!90!black},
    columns=flexible, numbers=left, numberstyle=\footnotesize, frame=tb, breakatwhitespace=false
}


% 可选:设置全局默认代码样式(按需启用)
% \lstset{style=MathematicaStyle}% 传进去后有 ../config/package.tex
    \begin{document}
\else% 如果定义了
\fi


\section{实数系的基本性质综述}


本节综述实数系的一些最基本的性质。这些性质将是我们以后讨论的基础。
以下分三组介绍这些性质:运算性质、顺序性质和连续性质。

\subsection*{一、运算性质}

在实数系 $\mathbb{R}$ 中定义了加法运算“$+$”和乘法运算“$\cdot$”,
使得对任意的 $a\in\mathbb{R}$ 和 $b\in\mathbb{R}$,
都有确定的 $a+b\in\mathbb{R}$ 和确定的 $a\cdot b\in\mathbb{R}$ 与之对应,
并且以下运算律成立:

\medskip
\begin{enumerate}[label=$(F_{\arabic*})$]
  \item 加法是交换的,即
  \[
    a+b=b+a,\qquad \forall\, a,b\in\mathbb{R}.
  \]

  \item 加法是结合的,即
  \[
    (a+b)+c=a+(b+c),\qquad \forall\, a,b,c\in\mathbb{R}.
  \]

  \item $0\in\mathbb{R}$ 对于加法起着特定的作用,即
  \[
    0+a=a+0=a,\qquad \forall\, a\in\mathbb{R}.
  \]

  \item 对每一个 $a\in\mathbb{R}$,都存在一个与它相反的数 $-a\in\mathbb{R}$,
  使得
  \[
    (-a)+a=a+(-a)=0.
  \]

  \item 乘法是交换的,即
  \[
    a\cdot b=b\cdot a,\qquad \forall\, a,b\in\mathbb{R}.
  \]

  \item 乘法是结合的,即
  \[
    (a\cdot b)\cdot c=a\cdot(b\cdot c),\qquad \forall\, a,b,c\in\mathbb{R}.
  \]

  \item $1\in\mathbb{R}$ 对于乘法起着特定的作用,即
  \[
    1\cdot a=a\cdot 1=a,\qquad \forall\, a\in\mathbb{R}.
  \]

  \item 对每一个 $a\in\mathbb{R}$ 且 $a\neq 0$,都存在一个倒数 $a^{-1}\in\mathbb{R}$,
  使得
  \[
    a^{-1}\cdot a=a\cdot a^{-1}=1.
  \]

  \item 乘法对于加法是分配的,即
  \[
    a\cdot(b+c)=a\cdot b+a\cdot c,\qquad \forall\, a,b,c\in\mathbb{R}.
  \]
\end{enumerate}

\subsection*{二、顺序性质}

在实数系 $\mathbb{R}$ 中定义了顺序关系“$<$”(
在以下的陈述中也出现记号“$>$”,我们约定:
“$a>b$”只是“$b<a$”的另一种写法,表示的是同一件事情)。
顺序关系“$<$”具有以下性质:

\medskip
\begin{enumerate}[label=($O_{\arabic*})$]
  \item 对任意的 $a\in\mathbb{R}$ 与 $b\in\mathbb{R}$,
  必有并且只有以下三种情形之一出现:
  \[
    a<b,\quad a=b,\quad \text{或者}\ a>b.
  \]
  (这一性质通常叫作\emph{三歧性}。)

  \item 关系“$<$”具有传递性,即
  \[
    a<b,\ b<c \;\Rightarrow\; a<c.
  \]

  \item 加以实数的运算保持顺序关系,即
  \[
    a<b \;\Rightarrow\; a+c<b+c.
  \]

  \item 乘以正实数的运算保持顺序关系,即
  \[
    a<b,\ c>0 \;\Rightarrow\; a\cdot c<b\cdot c.
  \]
\end{enumerate}

\subsection*{三、连续性质}

在实数系 $\mathbb{R}$ 中,以下的确界原理成立:

\begin{quote}
\textbf{(C) 确界原理}:
$\mathbb{R}$ 的任何一个非空而有上界的子集合,在 $\mathbb{R}$ 中都有上确界。
\end{quote}

我们对上面所列的性质做一些说明。

定义有加法与乘法运算并且符合运算律 $(F_1)\sim(F_9)$ 的集合,
通常称为\emph{域}。
实数系是一个域,有理数系和复数系也都是域。

定义有顺序关系“$<$”并且符合 \((O_1)\sim(O_4)\) 要求的一个域,
被称为\emph{有序域}。
实数系是一个有序域,有理数系也是一个有序域,
但复数系不是有序域。

确界原理 (C) 说明了实数系的连续性。
因此我们说:实数系 $\mathbb{R}$ 是一个\emph{连续的有序域}。



\ifx\allfiles\undefined %若没定义主文件
\end{document}% 就打印这个结尾
\fi