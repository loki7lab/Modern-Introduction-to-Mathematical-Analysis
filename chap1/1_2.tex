\ifx\allfiles\undefined% 如果这个没定义主文件
    \documentclass[12pt,a4paper,oneside]{ctexbook}
    \def\basepath{../}%basepath =  '../config'
    % -------------------------- 1. 路径与外部配置 --------------------------
% 定义基础路径变量(未提前定义则默认当前目录)
\ifx\basepath\undefined
    \newcommand{\basepath}{./}
\fi
% 引入外部宏包配置文件(按基础路径拼接)
% config/package.tex 
% 列出模板加载的包清单


% 基础数学和工具
\usepackage{amsmath} 
\usepackage{amsthm} 
\usepackage{amssymb}
\usepackage{mathrsfs} 
\usepackage{esint} 
\usepackage{yhmath} 
\usepackage{extarrows}
\usepackage{indentfirst}

% 颜色 (XColor 必须在 hyperref 之前)
\usepackage[dvipsnames, svgnames]{xcolor}
\usepackage[svgnames]{xcolor}
\usepackage{ulem} % 替代 soul



% 字体设置(fontspec 应该在 CJK 字体设置之前)
\usepackage{fontspec} 
\setCJKmainfont{Songti SC} 
\setmonofont{Courier New}
\setmainfont{Times New Roman}


% 布局和样式
\usepackage{geometry}
\usepackage{graphicx} 
\usepackage{fancyhdr} 
\usepackage{enumitem}
\usepackage[strict]{changepage} 
\usepackage{framed} 
\usepackage{listings}
% listings 依赖 xcolor,顺序没问题

\usepackage{setspace}  
\usepackage{microtype} 
\usepackage{titlesec}  

% 目录定制 (tocloft 必须在 hyperref 之前)
\usepackage{tocloft} 

% 超链接 (hyperref 必须是最后一个主要的宏包)
\usepackage{hyperref}

% 设置图片文件根路径(基础路径下的figure文件夹)
\graphicspath{{\basepath figure/}}


% -------------------------- 2. 页面尺寸与边距 --------------------------
\geometry{ 
    papersize={210mm,285mm}, % 页面尺寸(接近A4,略短)
    left=22mm, right=22mm,   % 左右边距
    top=28mm, bottom=28mm,   % 上下边距
    headheight=14pt, headsep=20pt % 页眉高度与间距
}


% -------------------------- 3. 章节标题样式 --------------------------
% 部分(Part)标题:新页+居中+特大号加粗
\titleformat{\part}{\newpage\centering\Huge\bfseries}{}{0em}{}
\titlespacing*{\part}{0pt}{0ex}{3.0ex} % 部分标题间距

% 章节(Chapter)标题:居中+大号加粗+带章节号
\titleformat{\chapter}{\centering\Large\bfseries}{\thechapter.}{1em}{}
\titlespacing*{\chapter}{0pt}{3.0ex}{2.0ex} % 章节标题间距

% 小节(Section)标题:大号加粗+带§符号
\titleformat{\section}{\centering\large\bfseries}{§\arabic{section}}{0.8em}{}
\titlespacing*{\section}{0pt}{2.5ex}{1.5ex} % 小节标题间距

% 自定义未编号章节(Explanation):封装chapter*,复用其样式
\newcommand{\explanation}[1]{\chapter*{#1}}



% -------------------------- 4. 段落与列表格式 --------------------------
\parindent=2em % 段落首行缩进2字符

% 有序/无序/描述列表统一格式:调整间距
\setenumerate[1]{itemsep=5pt, partopsep=0pt, parsep=\parskip, topsep=5pt}
\setitemize[1]{itemsep=5pt, partopsep=0pt, parsep=\parskip, topsep=5pt}
\setdescription{itemsep=5pt, partopsep=0pt, parsep=\parskip, topsep=5pt}

% 重新定义 \hl 命令
\definecolor{lightpink}{rgb}{1.0, 0.71, 0.76}
\newcommand{\hl}[1]{%
  \bgroup
  \markoverwith{\textcolor{lightpink}{\rule[-.5ex]{2pt}{2.5ex}}}%
  \ULon{#1}%
}


% -------------------------- 5. 页眉页脚配置 --------------------------
% -------------------------- 5. 页眉页脚配置 --------------------------
\setlength{\headheight}{15pt} % 解决 headheight too small 警告
\pagestyle{fancy} % 使用fancy样式
\fancyhf{} % 清空默认内容
\fancyhead[L]{\leftmark} % 左页眉:当前章节标题
\fancyhead[R]{\thepage} % 右页眉:页码
\renewcommand{\headrulewidth}{0.4pt} % 页眉下边框线宽
\renewcommand{\footrulewidth}{0pt} % 隐藏页脚下边框


% -------------------------- 6. 代码块样式(分语言定义) --------------------------
% 定义Mathematica代码样式(避免全局覆盖)
\lstdefinestyle{MathematicaStyle}{
    language=Mathematica, basicstyle=\tt, breaklines=true,
    keywordstyle=\bfseries\color{NavyBlue}, emphstyle=\bfseries\color{Rhodamine},
    commentstyle=\itshape\color{black!50!white}, stringstyle=\bfseries\color{PineGreen!90!black},
    columns=flexible, numbers=left, numberstyle=\footnotesize, frame=tb, breakatwhitespace=false
}

% 定义TeX代码样式(避免全局覆盖)
\lstdefinestyle{TeXStyle}{
    language=TeX, basicstyle=\ttfamily, breaklines=true,
    keywordstyle=\bfseries\color{NavyBlue}, emphstyle=\bfseries\color{Rhodamine},
    commentstyle=\itshape\color{black!50!white}, stringstyle=\bfseries\color{PineGreen!90!black},
    columns=flexible, numbers=left, numberstyle=\footnotesize, frame=tb, breakatwhitespace=false
}


% 可选:设置全局默认代码样式(按需启用)
% \lstset{style=MathematicaStyle}% 传进去后有 ../config/package.tex
    \begin{document}
\else% 如果定义了
\fi

\section{实数系的连续性}


关于实数系的连续性,有若干种相互等价的描述办法。
本节将要介绍的“确界原理”,就是其中便于运用的一种陈述方式。
通过在以后各章中的运用,读者将会逐渐加深对这一原理的理解。
先来介绍有关的术语。

\textbf{【上界与下界,有界集】}

设$E \subset \mathbb{R}$,$E \neq \varnothing$。
如果存在$L \in \mathbb{R}$(一个实数),使得  
$$x \leq L, \forall x \in E,$$

那么我们就说集合$E$有上界,并且说$L$是集合$E$的一个上界。
如果存在$l \in \mathbb{R}$,使得  
$$x \geq l, \forall x \in E,$$  
那么我们就说集合$E$有下界,并且说$l$是集合$E$的一个下界。
如果一个集合有上界并且也有下界,那么我们就说这集合有界,或者说这集合是有界集。

如果$L$是集合$E$的上界,$L_1 > L$,
那么$L_1$也是集合$E$的一个上界。
因此,一个有上界的集合,不可能有最大的上界(因为总能找到更大$L_n$)。
下面,我们来考察一个意义十分重大的问题:

$\implies$ 非空而有上界的实数集合,是否总有一个最小的上界?


\hl{这种“最小的上界”,通常称为上确界。}


\textbf{【上确界】}

设$E$是实数的非空集合,即设$E \subset \mathbb{R}$,$E \neq \varnothing$。
如果存在一个实数$M$,满足下面的条件(i)和(ii),那么我们就把$M$叫作集合$E$的上确界。
条件(i)和(ii)分别是:

(i) $M$是集合$E$的一个上界,即  
$x \leq M, \forall x \in E;$  

(ii) $M$是集合$E$的最小的上界:任何小于$M$的实数$M'$都不再是集合$E$的上界,即  
$(\forall M' < M)(\exists x' \in E)(x' > M').$  

上确界定义中的条件(ii)等价于说:集合$E$的任何上界$M_1 \geq M$。

如果$M$和$M_1$都是集合$E$的上确界,那么就应该有  
$M_1 \geq M, M \geq M_1,$  
因而有  
$M_1 = M,$  
由此得知:集合$E$的上确界如果存在就必定只有一个。我们把这唯一的上确界记为$$\sup E.$$

类似地可以定义下确界。

\textbf{【下确界】}

设$E \subset \mathbb{R}$,$E \neq \varnothing$。
如果存在一个实数$m$,满足以下的条件(i)和(ii),那么我们就把$m$叫作集合$E$的下确界:  

(i) $m$是集合$E$的一个下界,即  
$x \geq m, \forall x \in E;$  

(ii) $m$是集合$E$的最大的下界:任何大于$m$的实数$m'$都不再是集合$E$的下界,即  
$(\forall m' > m)(\exists x' \in E)(x' < m').$  

集合$E$的下确界如果存在就必定是唯一的。我们把这唯一的下确界记为$$\inf E.$$


设$E$是实数的非空集合。
我们以$-E$表示$E$中各数的相反数组成的集合,即定义  
$$-E = \{-x \mid x \in E\}.$$  

请读者自己验证以下简单事项:  

\begin{enumerate}
  \item 集合$E$有上界(下界)的充要条件是集合$-E$有下界(上界);
  \item 集合$E$有上确界的充要条件是集合$-E$有下确界,并且$\sup E = -\inf(-E)$;
  \item 集合$E$有下确界的充要条件是集合$-E$有上确界,并且$\inf E = -\sup(-E)$。
\end{enumerate}

证明示范:

$(\forall x \in E)\leq M \iff \forall x \in E , -x \geq -M \iff \forall y \in -E ,y\geq -M\text{(令y=-x)} \iff -E\text{有下界且为}-M.$

我们来介绍实数系的一个重要性质——\hl{连续性}。这一性质体现为以下的确界原理:  

\textbf{【确界原理(第一种陈述)】}

$\mathbb{R}$的任何非空而有上界的子集合$D$在$\mathbb{R}$中有上确界。  

我们将证明与这陈述等价的另一陈述(即证明的是第二种陈述):  

\textbf{【确界原理(第二种陈述)】} 

$\mathbb{R}$的任何一个非空并且有下界的子集合$E$在$\mathbb{R}$中有下确界。

证明:
在下面的讨论中,为了书写方便而作这样的约定:允许用记号\[\frac{1}{10^n}\]
代表相应的有尽小数
\[\underbrace{0.0\cdots 0}_\text{n个0}1  = \underbrace{0.0\cdots0}_\text{n个0}1000\cdots\]

我们分两种情形讨论。

\textbf{情形1:}  设0是集合$E$的一个下界。

因为$E\neq \varnothing$,所以$\exists x\in E$。
于是又$\exists k\in \mathbb{N}$,使得$k>x$。
我们看到:0是$E$的一个下界,$k$不是$E$的下界。

依次考察$0, 1, \dots, k-1$这些数,可以断定:
存在$a_0\in \{0,1,\dots,k-1\}$,
使得$a_0$是$E$的一个下界,而$a_0+1$不是$E$的下界。

然后依次考察$a_0.0, a_0.1, \dots, a_0.9$这些数,
又可断定:存在$a_1\in \{0,1,\dots,9\}$,
使得$a_0.a_1$是$E$的一个下界,而$a_0.a_1+\frac{1}{10}$不是$E$的下界。

再依次考察$a_0.a_{1}0,a_0.a_{1}1, \dots ,a_0.a_{1}9$这些数,又可断定:
$\exists a_2\in \{0,1,\dots,9\}$,
使得$a_0.a_1a_2$是$E$的一个下界,而$a_0.a_1a_2+\frac{1}{10^2}$不是$E$的下界。
继续这样做下去,我们得到一串数:
\begin{center}
$a_0, a_0.a_1, a_0.a_1a_2, \dots, a_0.a_1a_2\cdots a_n,\cdots$
\end{center}

这些数满足条件:
$a_0.a_1a_2\cdots a_n$是集合$E$的下界,
而$a_0.a_1a_2\cdots a_{n}+\frac{1}{10^n}$不是集合$E$的下界。

$\implies$ 现在需要证明的问题转变为:这样找出的$a_0.a_1a_2\cdots a_n\cdots$
是一个规范小数,且它正好就是集合$E$的下确界。

首先证明它是规范小数。使用反证法,假如$a_0.a_1a_2\cdots a_n$不是规范小数,
那么必定存在$p\in \mathbb{Z}^+$,使得
\[a_p+1 = a_p + 2 = \dots = 9\]
不妨设$p$是满足这个条件的最小的非负整数(从第$p+1$位开始,全是9)。
对任意的$\beta\in E$
,设$\beta$的规范小数表示为$\beta = \beta_0.\beta_1\beta_2\cdots$,
则必定存在$n>p$,使得$\beta_n < 9$(描述的是:因为$p<n$,所以此时$\alpha$的第$n$位是9,
$\beta$的第$n$位小于9)。
又因为上面的找法是找下界,因此\[\beta \geq a_0.a_1\cdots a_n\],
也就是必定存在$q \in \{0, 1, \dots, p\}$,使得

\begin{center}

$\beta_0 = a_0,\cdots ,\beta_{q-1} = a_{q-1},\beta_q \geq \alpha_q+1$
\end{center}
否则从第$p+1$位开始就没有希望比赢大小了。于是有
\begin{align*}
\beta &\geq a_0.a_1\cdots a_{q-1}(a_q+1)\\
        &\geq a_0.a_1\cdots a_{p-1}(a_p+1) \text{(这里是把位数往后放宽)}\\
        &=  a_0.a_1\cdots a_{p-1}a_p+\frac{1}{10^p}
\end{align*}
我们看到,
\begin{center}
$\beta \geq a_0.a_1\cdots a_p + \frac{1}{10^p}, \forall \beta \in E$
\end{center}

由$a_0.a_1\cdots$是非规范小数的假定,
导出$a_0.a_1\cdots a_p + \frac{1}{10^p}$也是下界,
而这与前面的选择办法相矛盾。
由此得知:$a_0.a_1\cdots a_n$必定是规范小数。\\

现在我们来证明这样找出的实数$a = a_0.a_1a_2\cdots$是集合$E$的下确界。
首先指出:$\forall \gamma \in E$必定满足
\begin{center}
$\gamma \geq a_0.a_1a_2\cdots .$
\end{center}
如果不是这样,就必定存在$h \in \mathbb{Z}^+$,
使得
\begin{center}
$\gamma < a_0.a_1a_2\cdots a_h$
\end{center}
,这与$a_0.a_1a_2\cdots a_h$的选取方法矛盾。
其次,对于任何一个$b > a_0.a_1a_2\cdots$,
必定存在$k \in \mathbb{Z}^+$,使得
\begin{center}
$b \geq a_0.a_1a_2\cdots a_k+ \frac{1}{10^k}.$
\end{center}

也就是哪怕$b$只要比前面算法找到的大一点,也会导致$a_0.a_1a_2\cdots a_k+ \frac{1}{10^k}$成为下界,
因此$b$不可能是集合$E$的下界。
至此,对于0是$E$的下界的情形,我们证明了集合$E$在$\mathbb{R}$中必定有下确界。\\


\textbf{情形2:} 设0不是集合$E$的下界。

这就是说,
\begin{center}
$\exists x \in E \text{ such that } x < 0.$
\end{center}
于是,$E$的任何下界$l$必定小于0:
\begin{center}
$l < 0.$
\end{center}
我们来考察$\mathbb{R}$的另一非空子集合
\begin{center}
$F = \{-l \mid l \text{ 是 } E \text{ 的下界}\}$
\end{center}
。容易看出:0是集合$F$的一个下界。利用情形1中已经证明的结果,
可以断定:$F$在$\mathbb{R}$中有下确界,即
\begin{center}
$\exists c = \inf F \in \mathbb{R}.$
\end{center}
我们指出:$a = -c$是集合$E$的下确界。\\
为此,考察$\gamma \in E$。
显然对任何$-l \in F$都有
\begin{center}
$\gamma \geq l$,$-\gamma \leq -l$
\end{center}
这说明$-\gamma$是集合$F$的一个下界。因而,
\begin{center}
$-\gamma \leq c$,$\gamma \geq -c = a$
\end{center}
。这说明$a = -c$是集合$E$的一个下界。
另一方面,对于任意的$b > a$,我们有$-b < -a = c$,
因此$-b \notin F$。
这就是说,任何大于$a$的实数$b$都不是集合$E$的下界。
我们证明了$a$是$E$的下确界。$\square$



\ifx\allfiles\undefined %若没定义主文件
\end{document}% 就打印这个结尾
\fi