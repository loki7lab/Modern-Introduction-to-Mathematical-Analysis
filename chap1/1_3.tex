\ifx\allfiles\undefined% 如果这个没定义主文件
    \documentclass[12pt,a4paper,oneside,utf8]{ctexbook}
    \def\basepath{../}%basepath =  ^{'} ../config^{'} 
    % -------------------------- 1. 路径与外部配置 --------------------------
% 定义基础路径变量(未提前定义则默认当前目录)
\ifx\basepath\undefined
    \newcommand{\basepath}{./}
\fi
% 引入外部宏包配置文件(按基础路径拼接)
% config/package.tex 
% 列出模板加载的包清单


% 基础数学和工具
\usepackage{amsmath} 
\usepackage{amsthm} 
\usepackage{amssymb}
\usepackage{mathrsfs} 
\usepackage{esint} 
\usepackage{yhmath} 
\usepackage{extarrows}
\usepackage{indentfirst}

% 颜色 (XColor 必须在 hyperref 之前)
\usepackage[dvipsnames, svgnames]{xcolor}
\usepackage[svgnames]{xcolor}
\usepackage{ulem} % 替代 soul



% 字体设置(fontspec 应该在 CJK 字体设置之前)
\usepackage{fontspec} 
\setCJKmainfont{Songti SC} 
\setmonofont{Courier New}
\setmainfont{Times New Roman}


% 布局和样式
\usepackage{geometry}
\usepackage{graphicx} 
\usepackage{fancyhdr} 
\usepackage{enumitem}
\usepackage[strict]{changepage} 
\usepackage{framed} 
\usepackage{listings}
% listings 依赖 xcolor,顺序没问题

\usepackage{setspace}  
\usepackage{microtype} 
\usepackage{titlesec}  

% 目录定制 (tocloft 必须在 hyperref 之前)
\usepackage{tocloft} 

% 超链接 (hyperref 必须是最后一个主要的宏包)
\usepackage{hyperref}

% 设置图片文件根路径(基础路径下的figure文件夹)
\graphicspath{{\basepath figure/}}


% -------------------------- 2. 页面尺寸与边距 --------------------------
\geometry{ 
    papersize={210mm,285mm}, % 页面尺寸(接近A4,略短)
    left=22mm, right=22mm,   % 左右边距
    top=28mm, bottom=28mm,   % 上下边距
    headheight=14pt, headsep=20pt % 页眉高度与间距
}


% -------------------------- 3. 章节标题样式 --------------------------
% 部分(Part)标题:新页+居中+特大号加粗
\titleformat{\part}{\newpage\centering\Huge\bfseries}{}{0em}{}
\titlespacing*{\part}{0pt}{0ex}{3.0ex} % 部分标题间距

% 章节(Chapter)标题:居中+大号加粗+带章节号
\titleformat{\chapter}{\centering\Large\bfseries}{\thechapter.}{1em}{}
\titlespacing*{\chapter}{0pt}{3.0ex}{2.0ex} % 章节标题间距

% 小节(Section)标题:大号加粗+带§符号
\titleformat{\section}{\centering\large\bfseries}{§\arabic{section}}{0.8em}{}
\titlespacing*{\section}{0pt}{2.5ex}{1.5ex} % 小节标题间距

% 自定义未编号章节(Explanation):封装chapter*,复用其样式
\newcommand{\explanation}[1]{\chapter*{#1}}



% -------------------------- 4. 段落与列表格式 --------------------------
\parindent=2em % 段落首行缩进2字符

% 有序/无序/描述列表统一格式:调整间距
\setenumerate[1]{itemsep=5pt, partopsep=0pt, parsep=\parskip, topsep=5pt}
\setitemize[1]{itemsep=5pt, partopsep=0pt, parsep=\parskip, topsep=5pt}
\setdescription{itemsep=5pt, partopsep=0pt, parsep=\parskip, topsep=5pt}

% 重新定义 \hl 命令
\definecolor{lightpink}{rgb}{1.0, 0.71, 0.76}
\newcommand{\hl}[1]{%
  \bgroup
  \markoverwith{\textcolor{lightpink}{\rule[-.5ex]{2pt}{2.5ex}}}%
  \ULon{#1}%
}


% -------------------------- 5. 页眉页脚配置 --------------------------
% -------------------------- 5. 页眉页脚配置 --------------------------
\setlength{\headheight}{15pt} % 解决 headheight too small 警告
\pagestyle{fancy} % 使用fancy样式
\fancyhf{} % 清空默认内容
\fancyhead[L]{\leftmark} % 左页眉:当前章节标题
\fancyhead[R]{\thepage} % 右页眉:页码
\renewcommand{\headrulewidth}{0.4pt} % 页眉下边框线宽
\renewcommand{\footrulewidth}{0pt} % 隐藏页脚下边框


% -------------------------- 6. 代码块样式(分语言定义) --------------------------
% 定义Mathematica代码样式(避免全局覆盖)
\lstdefinestyle{MathematicaStyle}{
    language=Mathematica, basicstyle=\tt, breaklines=true,
    keywordstyle=\bfseries\color{NavyBlue}, emphstyle=\bfseries\color{Rhodamine},
    commentstyle=\itshape\color{black!50!white}, stringstyle=\bfseries\color{PineGreen!90!black},
    columns=flexible, numbers=left, numberstyle=\footnotesize, frame=tb, breakatwhitespace=false
}

% 定义TeX代码样式(避免全局覆盖)
\lstdefinestyle{TeXStyle}{
    language=TeX, basicstyle=\ttfamily, breaklines=true,
    keywordstyle=\bfseries\color{NavyBlue}, emphstyle=\bfseries\color{Rhodamine},
    commentstyle=\itshape\color{black!50!white}, stringstyle=\bfseries\color{PineGreen!90!black},
    columns=flexible, numbers=left, numberstyle=\footnotesize, frame=tb, breakatwhitespace=false
}


% 可选:设置全局默认代码样式(按需启用)
% \lstset{style=MathematicaStyle}% 传进去后有 ../config/package.tex
    \begin{document}
\else% 如果定义了
\fi

\section{实数的四则运算}

两个实数的和、差、积、商是什么意思?这是需要予以确切定义的。
为了定义实数 $a$ 与 $b$ 之和,我们考察满足以下条件的有尽小数
$\alpha, \alpha^{'}  $ 与 $\beta, \beta^{'}  $:
\[
\alpha \le a \le \alpha^{'}  , \qquad \beta \le b \le \beta^{'}  .
\]
两实数 $a$ 与 $b$ 之和 $a+b$ 的合理定义应该满足
\[
\alpha+\beta \le a+b \le \alpha^{'}  +\beta^{'}  .
\]
上式中的 $\alpha+\beta$ 和 $\alpha^{'}  +\beta^{'}  $ 都只涉及有尽小数的加法运算,
因而是已经有定义的。我们将利用已有定义的有尽小数的运算来定义实数的相应运算。
(接下来请注意区分定理,引理的关系。)

\medskip

\hlpink{\textbf{定理 1}}\quad
设 $a$ 和 $b$ 是实数(不要求有尽),则存在唯一实数 $u$(不要求有尽),使得对于满足条件
\[
\alpha \le a \le \alpha^{'}  , \qquad \beta \le b \le \beta^{'}  
\]
的\hlgreen{任何有尽小数} $\alpha, \alpha^{'}  $ 和 $\beta, \beta^{'}  $,都有
\[
\alpha+\beta \le u \le \alpha^{'}  +\beta^{'}  .
\]

证明这样的实数$u$存在比较容易。事实上,实数

\[
u = \sup \left\{ \alpha + \beta \;\middle|\;
\begin{aligned}
& \alpha \text{ 和 } \beta \text{ 是有尽小数,} \\
& \alpha \leqslant a, \beta \leqslant b
\end{aligned}
\right\}
\]
就符合定理的要求。唯一性的证明基于以下想法:我们可以取彼此充分靠近的有尽小数
$\alpha, \alpha^{'}  $,以及彼此充分靠近的有尽小数 $\beta, \beta^{'}  $,使得
\[
\alpha \le a \le \alpha^{'}  , \qquad \beta \le b \le \beta^{'}  .
\]
于是 $\alpha+\beta$ 与 $\alpha^{'}  +\beta^{'}  $ 可以任意接近,
因而在它们之间容不下两个不同的数(Sandwich / Squeeze Theorem)。

以上推想方式是令人信服的,但要严格地写出每一步证明,却是一件细致的工作。
我们把这部分内容放到本节后的附录中,供喜欢寻根究底的读者参考。
初学者不必也不宜在这些细节上花费太多时间,尤其不要因此而分散了对主要问题的注意力。

\medskip
\textbf{定义 1}\quad
我们把定理 1 中所述的唯一确定的实数 $u$ 叫作实数 $a$ 与实数 $b$ 之和,
并约定把它记为 $a+b$。

\medskip
\textbf{定义 2}\quad
实数 $a$ 与实数 $b$ 之差定义为 $a$ 与 $-b$ 之和,即规定
\[
a-b = a+(-b).
\]

\medskip
为了定义两个非负实数的乘积,我们需要以下定理。

\medskip
\hlpink{\textbf{定理 2}}\quad
设 $a$ 和 $b$ 是非负实数(不要求有尽),则存在唯一实数$v$(不要求有尽),
使得对于满足条件
\[
0 \le \alpha \le a \le \alpha^{'}  , \qquad 0 \le \beta \le b \le \beta^{'}  
\]
的\hlgreen{任何有尽小数} $\alpha, \alpha^{'}  $ 和 $\beta, \beta^{'}  $,都有
\[
\alpha \beta \le v \le \alpha^{'}   \beta^{'}  .
\]

这一定理的证明也放在本节后的附录中。

\medskip
\textbf{定义 3}\quad
我们把定理 2 中所述的唯一确定的实数 $v$ 叫作非负实数 $a$ 与非负实数 $b$
的乘积,并约定把它记为 $ab$。

\medskip
\textbf{定义 4}\quad
任意实数 $a$ 与 $b$ 的乘积 $ab$ 定义如下:
\[
ab =
\begin{cases}
|a|\,|b|, & \text{如果 } a \text{ 与 } b \text{ 同号},\\[0.3em]
-|a|\,|b|, & \text{如果 } a \text{ 与 } b \text{ 异号}.
\end{cases}
\]

至于实数的除法,我们将在下面的附录中予以讨论。


%%%%%%%%%%%%%%%%%%%%%%%%%%%%%%%%%%%%%%%%%%%%%%%%%%%%%%%%%%%%
\medskip
\begin{center}
补充内容
\end{center}

在这部分内容里,我们补充定理1 和定理2 的证明,并对实数的除法作相应的讨论。

\medskip
\textbf{引理 1}\quad
设 $a$ 是任意一个实数(不要求有尽),
则对\hlgreen{任何正的有尽小数 $\varepsilon$},
\hlgreen{存在有尽小数$\alpha$ 和 $\alpha^{'}  $},满足条件
\[
\alpha \le a \le \alpha^{'}  , \qquad \alpha^{'}   - a < \varepsilon .
\]

(无论实数本身是不是有尽,它总能被两个足够近的有尽小数描述)

\medskip
\textbf{证明}\quad
我们设
\[
\varepsilon = \varepsilon_0.\varepsilon_1 \cdots \varepsilon_p ,
\]
并设其中第一位不等于 $0$ 的数字出现在第 $k-1$位,
其中 $0 \le k-1 \le p$,则有
\[
\frac{1}{10^k} < \varepsilon .
\]

若 $a$ 的规范小数表示为
\[
a = a_0.a_1 a_2 \cdots ,
\]
则(根据$\frac{1}{10^k} < \varepsilon$)取
\[
\alpha = a_0.a_1 \cdots a_k, \qquad
\alpha^{'}   = a_0.a_1 \cdots a_k + \frac{1}{10^k}.
\]

(即:找出规范小数$a$的第$k$位小数,
$\alpha$等于$a$的前$k$位,$\alpha^{'}  $其余部分一样,第$k$位小数要$+1$)

若 $a$ 的规范小数表示为
\[
a = -a_0.a_1 a_2 \cdots ,
\]
则取
\[
\alpha = -a_0.a_1 \cdots a_k - \frac{1}{10^k}, \qquad
\alpha^{'}   = -a_0.a_1 \cdots a_k .
\]

对这两种情形都有
\[
\alpha \le a \le \alpha^{'}  , \qquad
\alpha^{'}   - a = \frac{1}{10^k} < \varepsilon .
\]
证毕。\qed

\medskip
\textbf{引理 2}\quad
设 $c$ 和 $c^{'}  $ 是实数,且 $c \le c^{'}  $。
如果对\hlgreen{任何正的有尽小数 $\varepsilon$},
\hlgreen{存在有尽小数 $\gamma$ 和 $\gamma^{'}  $},满足条件
\[
\gamma \le c \le c^{'}   \le \gamma^{'}  , \qquad
\gamma^{'}   - \gamma < \varepsilon ,
\]
那么必定有
\[
c = c^{'}   .
\]

(当一个实数不小于另一个实数(无论这两个实数是不是有尽),
能够落入一对足够近的有尽实数内,那么这两个实数相等。)

\medskip
\textbf{证明}\quad
用反证法。假如 $c \neq c^{'}   \implies c < c^{'}  $,那么存在有尽小数 $\eta$ 和 $\eta^{'}  $,使得
\[
c < \eta < \eta^{'}   < c^{'}   .
\]
对于$\varepsilon = \eta^{'}   - \eta > 0$(此时$\varepsilon$是有尽的),任何满足
\[
\gamma \le c < \eta < \eta^{'}   < c^{'}   \le \gamma^{'}  
\]
的有尽小数 $\gamma$ 和 $\gamma^{'}  $,都有
\[
\gamma^{'}   - \gamma \ge \eta^{'}   - \eta = \varepsilon ,
\]
这与 $\gamma^{'}   - \gamma < \varepsilon$ 矛盾。

因此,如果引理所述的前提成立,则必定有 $c = c^{'}  $。
证毕。\qed

\medskip
\textbf{引理 3}\quad
设 $\varepsilon$ 是正的有尽小数,$M$ 和 $N$ 是自然数,
则存在正的有尽小数 $\varepsilon^{'}  $ 和 $\varepsilon^{''}  $,使得
\[
M \varepsilon^{'}  + N \varepsilon^{''}   < \varepsilon .
\]
(总能有一对正的有尽小数,能把两个自然数关进一个足够小的区间)

\medskip
\textbf{证明}\quad
我们设
\[
\varepsilon = \varepsilon_0.\varepsilon_1 \cdots \varepsilon_p ,
\]
并设其中第一位不等于 $0$ 的数字是 $\varepsilon_{k-1}$,
其中 $0 \le k-1 \le p$,则有
\[
\frac{1}{10^k} < \varepsilon .
\]

对引理中$M,N$,我们取自然数 $m,n$,使得
\[
10^m \ge M, \qquad 10^n \ge N .
\]
再取
\begin{align*}
\varepsilon^{'}   = \frac{1}{10^{m+k+1}}, \qquad
\varepsilon^{''}   = \frac{1}{10^{n+k+1}}   
\end{align*}

于是
\begin{align*}
M\varepsilon^{'}   + N\varepsilon^{''}   &\leq 10^m \varepsilon^{'}   + 10^n \varepsilon^{''}  \\
        &= \frac{1}{10^{k+1}} + \frac{1}{10^{k+1}}\\
        &< \frac{1}{10^k}\\
        &< \varepsilon. \qed
\end{align*}

现在开始证明前面正文的内容。

\medskip
\hlgreen{\textbf{定理 1 的证明}}

\textbf{存在性}\quad
实数\[
u = \sup \left\{ \alpha + \beta \;\middle|\;
\begin{aligned}
& \alpha \text{ 和 } \beta \text{ 是有尽小数,} \\
& \alpha \leqslant a, \beta \leqslant b
\end{aligned}
\right\}
\] 符合定理的要求。

\medskip
\textbf{唯一性}\quad
对于任意正的有尽小数 $\varepsilon$和自然数 $M=N=1$。
根据引理3,存在正的有尽小数 $\varepsilon^{'}  $ 和 $\varepsilon^{''}  $,
使得
\[
\varepsilon^{'}   + \varepsilon^{''}  < \varepsilon .
\]

又根据引理1,存在有尽小数 $\alpha,\alpha^{'}  $ 和 $\beta,\beta^{'}  $,
分别满足
\[
\alpha \le a \le \alpha^{'}  , \qquad \alpha^{'}   - a < \varepsilon^{'}  ,
\]
以及
\[
\beta \le b \le \beta^{'}  , \qquad \beta^{'}   - \beta < \varepsilon^{''}   .
\]

于是有
\[
(\alpha^{'}   + \beta^{'}  ) - (\alpha + \beta) < 2\varepsilon .
\]
(前面的2不影响数量级)由于 $\varepsilon$ 可以取任意正的有尽小数(引理2:意味着可以把那两个实数关进$\implies$ 这实际是一个数),
根据引理2,满足
\[
\alpha + \beta \le u \le \alpha^{'}   + \beta^{'}  
\]
的实数 $u$ 是唯一的。\qed

\medskip
\hlgreen{\textbf{定理2的证明}}

\textbf{存在性}\quad
实数\[
z = \sup \left\{ \alpha\beta \;\middle|\;
\begin{aligned}
& \alpha \text{ 和 } \beta \text{ 是有尽小数,} \\
& 0\leq \alpha \leq a, 0\leq \beta \leq b
\end{aligned}
\right\}
\] 符合定理的要求。


\textbf{唯一性}\quad
首先取自然数 $M,N$,使得
\[
0 \leq a < M, \qquad 0 \le b < N .
\]
其次,对于任意正的有尽小数 $\varepsilon$,
根据引理3,存在正的有尽小数 $\varepsilon^{'} $ 和 $\varepsilon^{''} $,
使得
\[
M\varepsilon^{'}  + N\varepsilon^{''} < \varepsilon .
\]

又根据引理1,存在有尽小数 $\alpha,\alpha^{'} $ 和 $\beta,\beta^{'} $,
分别满足
\[
0 \le \alpha \le a \le \alpha^{'}  < M, \qquad \alpha^{'}  - \alpha < \varepsilon^{''}  ,
\]
以及
\[
0 \le \beta \le b \le \beta^{'}  < N, \qquad \beta^{'}  - \beta < \varepsilon^{'}  .
\]

于是
\begin{align*}
\alpha^{'} \beta^{'}  - \alpha\beta
&= \alpha^{'} \beta^{'}  - \alpha^{'} \beta + \alpha^{'} \beta - \alpha\beta\\
&= \alpha^{'} (\beta^{'}  - \beta) + (\alpha^{'}  - \alpha)\beta \\
&< M\varepsilon^{'}  + N\varepsilon^{''}   \\
&< \varepsilon .
\end{align*}

由于 $\varepsilon$ 可以取任意正的有尽小数,
根据引理2,符合定理要求的实数 $v$ 是唯一的。\qed

%%%%%%%%%%%%%%%%%%%%%%%%%%%%%%%%%%%%%%%%%%%%%%%%%%%%%%%%%%%
下面讨论实数的\hlgreen{除法}。在初等数学的课程里,我们学习过有尽小数的“长除法”(除法竖式)。
这是一种可以用来确定近似商的除法手段。
对于给定的正的有尽小数 $\alpha,\beta$ 和自然数 $n$,
通过逐位试商,可以确定存在一个有尽小数
\[
\gamma = \gamma_0.\gamma_1 \cdots \gamma_n
\]
满足
\[
\gamma\cdot \alpha \le \beta < \left(\gamma+\frac{1}{10^n}\right)\cdot \alpha .
\]
对于给定的 $\alpha,\beta$ 和 $n$,这样的 $\gamma$ 是唯一确定的。
我们把这样的
\[
\gamma \quad \text{和} \quad \gamma^{'}  = \gamma+\frac{1}{10^n}
\]
分别叫作 $\beta \div \alpha$ 的、精确到小数点以后 $n$ 位的
\textbf{不足近似商}和\textbf{过剩近似商},
并约定用以下记号表示它们:
\[
(\frac{\beta}{\alpha})_n = \gamma, \qquad (\frac{\beta}{\alpha})_n^{'}  = \gamma ^{'} .
\]

为了定义任意实数 $b$ 除以任意非零实数 $a$ 的商,
可以先考察 $a>0,\; b=1$ 的情形。
只要对 $a>0$ 的情形定义了 $1/a$,
就可以按以下方式定义任意实数 $b$ 除以任意非零实数 $a$ 的商:
\[
\frac{b}{a} =
\begin{cases}
 \dfrac{1}{|a|}\cdot |b| , & a \text{ 与 } b \text{ 同号},\\[6pt]
-\dfrac{1}{|a|}\cdot |b|, & a \text{ 与 } b \text{ 异号}.
\end{cases}
\]


\hlpink{\textbf{定理 3}}\quad
对任何正实数 $a$,存在唯一的正实数 $w$,
使得对于满足
\[
0<\alpha \le a \le \alpha^{'} 
\]
的任何有尽小数 $\alpha,\alpha^{'} $,
以及任意自然数 $m,n$,都有
\[
\left(\frac{1}{\alpha^{'} }\right)_m \le w \le \left(\frac{1}{\alpha}\right)_n^{'}  .
\]

\medskip
\textbf{定义}\quad
我们把定理3中所述的唯一确定的正实数 $w$
叫作正实数 $a$ 的\textbf{倒数},
并把它记为 $1/a$。在下面的证明中,我们将利用有尽小数乘法的以下性质:

对于任何正的有尽小数 $\beta,\gamma,\gamma^{'} $,都有
\begin{align*}
\beta\gamma < \beta\gamma^{'}  &\iff \gamma < \gamma^{'} , \\
\beta\gamma = \beta\gamma^{'}  &\iff \gamma = \gamma^{'} , \\
\beta\gamma > \beta\gamma^{'}  &\iff \gamma > \gamma^{'} .
\end{align*}

\medskip
\hlgreen{\textbf{定理 3 的证明}}

\medskip
\textbf{存在性}\quad
由于
\[
\alpha^{'}  \cdot \left(\frac{1}{\alpha^{'} }\right)_m
\le 1
\le \alpha \cdot \left(\frac{1}{\alpha^{'} }\right)_n
\le \alpha^{'}  \cdot \left(\frac{1}{\alpha^{'} }\right)_n
\]
所以
\[
\left(\frac{1}{\alpha^{'} }\right)_m
\le \left(\frac{1}{\alpha}\right)_n^{'},
\qquad \forall\, m,n\in\mathbb{N}.
\]
由此容易看出,实数 
实数\[
w = \sup \left\{(\frac{1}{\alpha^{'}})_m \;\middle|\;
\begin{aligned}
& \alpha^{'} \text{是有尽小数,} \\
& \alpha^{'} \geq a \text{,} m \in \mathbb{N}
\end{aligned}
\right\}
\] 
符合定理的要求。

\medskip
\textbf{唯一性}\quad
首先,取
\[
\sigma=\frac{1}{10^k}, \qquad M=10^l,
\]
使得
\[
0<\sigma<a<M.
\]
其次,设 $\varepsilon$ 是任意一个正的有尽小数。
根据引理3,存在正的有尽小数 $\varepsilon^{'} ,\varepsilon^{''}  $,
使得
\[
\varepsilon^{'}  + 10^{2(k+1)}\varepsilon^{''}   < \varepsilon .
\]
从而
\[
\sigma^2\varepsilon^{'}  + M^2\varepsilon^{''}  \\
=\sigma^2(\varepsilon^{'}  + 10^{2(k+1)}\varepsilon^{''})\\
< \sigma^2\varepsilon .
\]

我们可以选取有尽小数 $\alpha,\alpha^{'} $ 和自然数 $n$,
满足
\[
0<\sigma<a\le \alpha^{'}  < M, \qquad
\alpha^{'}  - a < \sigma^2\varepsilon^{'} , \qquad
\frac{1}{10^{n-1}} < \varepsilon^{''}   .
\]
这样选取$\alpha$,$\alpha^{'}$和$n$应该使得



\begin{align*}
\sigma^2\left\{ \left( \frac{1}{\alpha} \right)_n^{'} -
 \left( \frac{1}{\alpha^{'}} \right)_n \right\} 
&\leq \alpha \alpha^{'}\left\{ \left( \frac{1}{\alpha} \right)_n^{'} - 
\left( \frac{1}{\alpha^{'}} \right)_n \right\} \\
&= \alpha \alpha^{'}\left\{ \left( \left( \frac{1}{\alpha} \right)_n + 
\frac{1}{10^n} \right) - \left( \left( \frac{1}{\alpha^{'}} \right)_n^{'} - 
\frac{1}{10^n} \right) \right\} \\
&= \alpha^{'}\left\{ \alpha \left( \frac{1}{\alpha} \right)_n \right\} - 
\alpha^{'}\left\{ \alpha \left( \frac{1}{\alpha^{'}} \right)_n^{'} \right\} + 
2\alpha \alpha^{'} \frac{1}{10^n} \\
&< \alpha^{'} - \alpha + \alpha \alpha^{'} \frac{1}{10^{n-1}} \\
&< \sigma^2 \varepsilon^{'} + M^2 \varepsilon^{''} < \sigma^2 \varepsilon.
\end{align*}

于是有
\[
(\frac{1}{\alpha})^{'}_n - (\frac{1}{\alpha})_n < \varepsilon
\]
因为这里的 $\varepsilon$ 可以是任意正的有尽小数,
故满足定理条件的实数 $w$ 不能多于一个,
从而唯一性得证。\qed


\ifx\allfiles\undefined %若没定义主文件
\end{document}% 就打印这个结尾
\fi