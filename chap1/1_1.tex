\ifx\allfiles\undefined% 如果这个没定义主文件
    \documentclass[12pt,a4paper,oneside]{ctexbook}
    \def\basepath{../}%basepath =  '../config'
    % -------------------------- 1. 路径与外部配置 --------------------------
% 定义基础路径变量(未提前定义则默认当前目录)
\ifx\basepath\undefined
    \newcommand{\basepath}{./}
\fi
% 引入外部宏包配置文件(按基础路径拼接)
% config/package.tex 
% 列出模板加载的包清单


% 基础数学和工具
\usepackage{amsmath} 
\usepackage{amsthm} 
\usepackage{amssymb}
\usepackage{mathrsfs} 
\usepackage{esint} 
\usepackage{yhmath} 
\usepackage{extarrows}
\usepackage{indentfirst}

% 颜色 (XColor 必须在 hyperref 之前)
\usepackage[dvipsnames, svgnames]{xcolor}
\usepackage[svgnames]{xcolor}
\usepackage{ulem} % 替代 soul



% 字体设置(fontspec 应该在 CJK 字体设置之前)
\usepackage{fontspec} 
\setCJKmainfont{Songti SC} 
\setmonofont{Courier New}
\setmainfont{Times New Roman}


% 布局和样式
\usepackage{geometry}
\usepackage{graphicx} 
\usepackage{fancyhdr} 
\usepackage{enumitem}
\usepackage[strict]{changepage} 
\usepackage{framed} 
\usepackage{listings}
% listings 依赖 xcolor,顺序没问题

\usepackage{setspace}  
\usepackage{microtype} 
\usepackage{titlesec}  

% 目录定制 (tocloft 必须在 hyperref 之前)
\usepackage{tocloft} 

% 超链接 (hyperref 必须是最后一个主要的宏包)
\usepackage{hyperref}

% 设置图片文件根路径(基础路径下的figure文件夹)
\graphicspath{{\basepath figure/}}


% -------------------------- 2. 页面尺寸与边距 --------------------------
\geometry{ 
    papersize={210mm,285mm}, % 页面尺寸(接近A4,略短)
    left=22mm, right=22mm,   % 左右边距
    top=28mm, bottom=28mm,   % 上下边距
    headheight=14pt, headsep=20pt % 页眉高度与间距
}


% -------------------------- 3. 章节标题样式 --------------------------
% 部分(Part)标题:新页+居中+特大号加粗
\titleformat{\part}{\newpage\centering\Huge\bfseries}{}{0em}{}
\titlespacing*{\part}{0pt}{0ex}{3.0ex} % 部分标题间距

% 章节(Chapter)标题:居中+大号加粗+带章节号
\titleformat{\chapter}{\centering\Large\bfseries}{\thechapter.}{1em}{}
\titlespacing*{\chapter}{0pt}{3.0ex}{2.0ex} % 章节标题间距

% 小节(Section)标题:大号加粗+带§符号
\titleformat{\section}{\centering\large\bfseries}{§\arabic{section}}{0.8em}{}
\titlespacing*{\section}{0pt}{2.5ex}{1.5ex} % 小节标题间距

% 自定义未编号章节(Explanation):封装chapter*,复用其样式
\newcommand{\explanation}[1]{\chapter*{#1}}



% -------------------------- 4. 段落与列表格式 --------------------------
\parindent=2em % 段落首行缩进2字符

% 有序/无序/描述列表统一格式:调整间距
\setenumerate[1]{itemsep=5pt, partopsep=0pt, parsep=\parskip, topsep=5pt}
\setitemize[1]{itemsep=5pt, partopsep=0pt, parsep=\parskip, topsep=5pt}
\setdescription{itemsep=5pt, partopsep=0pt, parsep=\parskip, topsep=5pt}

% 重新定义 \hl 命令
\definecolor{lightpink}{rgb}{1.0, 0.71, 0.76}
\newcommand{\hl}[1]{%
  \bgroup
  \markoverwith{\textcolor{lightpink}{\rule[-.5ex]{2pt}{2.5ex}}}%
  \ULon{#1}%
}


% -------------------------- 5. 页眉页脚配置 --------------------------
% -------------------------- 5. 页眉页脚配置 --------------------------
\setlength{\headheight}{15pt} % 解决 headheight too small 警告
\pagestyle{fancy} % 使用fancy样式
\fancyhf{} % 清空默认内容
\fancyhead[L]{\leftmark} % 左页眉:当前章节标题
\fancyhead[R]{\thepage} % 右页眉:页码
\renewcommand{\headrulewidth}{0.4pt} % 页眉下边框线宽
\renewcommand{\footrulewidth}{0pt} % 隐藏页脚下边框


% -------------------------- 6. 代码块样式(分语言定义) --------------------------
% 定义Mathematica代码样式(避免全局覆盖)
\lstdefinestyle{MathematicaStyle}{
    language=Mathematica, basicstyle=\tt, breaklines=true,
    keywordstyle=\bfseries\color{NavyBlue}, emphstyle=\bfseries\color{Rhodamine},
    commentstyle=\itshape\color{black!50!white}, stringstyle=\bfseries\color{PineGreen!90!black},
    columns=flexible, numbers=left, numberstyle=\footnotesize, frame=tb, breakatwhitespace=false
}

% 定义TeX代码样式(避免全局覆盖)
\lstdefinestyle{TeXStyle}{
    language=TeX, basicstyle=\ttfamily, breaklines=true,
    keywordstyle=\bfseries\color{NavyBlue}, emphstyle=\bfseries\color{Rhodamine},
    commentstyle=\itshape\color{black!50!white}, stringstyle=\bfseries\color{PineGreen!90!black},
    columns=flexible, numbers=left, numberstyle=\footnotesize, frame=tb, breakatwhitespace=false
}


% 可选:设置全局默认代码样式(按需启用)
% \lstset{style=MathematicaStyle}% 传进去后有 ../config/package.tex
    \begin{document}
\else% 如果定义了
\fi


\section{实数的无尽小数表示与顺序}

在初等数学课程里,我们已经熟悉了有尽小数,会做有尽小数的加减法和乘法运算。
我们还知道,任何有理数都可以表示为无尽循环小数(有尽小数看成后面接有一串0的无尽循环小数)。
在此基础上,我们进一步引入无尽不循环小数以表示无理数。
这样,一般地以无尽小数表示实数。以这种朴素的理解为背景,我们来考察实数的顺序,讨论实数系的连续性问题,
并定义实数的运算。


\textbf{【无尽小数】}  

形状如$$\pm a_0.a_1a_2\cdots a_n\cdots$$这样的表示被称为无尽小数,
这里$a_0 \in \mathbb{Z}^+$,而$a_1,a_2,\cdots,a_n,\cdots$中的每一个都是0,1,$\cdots$,9这些数字之一。
形状如$+a_0.a_1a_2\cdots a_n\cdots$的无尽小数常常简单地写为$a_0.a_1a_2\cdots a_n\cdots$。
我们还约定:形如$\pm a_0.a_1a_2\cdots a_m0000\cdots$这样的无尽小数可以写成$\pm a_0.a_1a_2\cdots a_m$,
并可称之为有尽小数。


\textbf{【等同关系】}

我们给无尽小数规定如下的等同关系$(E_1)$和$(E_2)$:
\begin{align*}
(E_1) \quad &-0.000\cdots = +0.000\cdots, \\
(E_2) \quad &\pm b_0.b_1\cdots b_p999\cdots = \pm b_0.b_1\cdots (b_p + 1)000\cdots \quad (\text{其中} \, b_p < 9).
\end{align*}

如同$(E_1)$和$(E_2)$两式中等号左边那样的无尽小数被称为非规范小数,其他的无尽小数都称为规范小数。
所规定的等同关系将每一个非规范小数等同于一个与它相对应的规范小数(注释:$b_p < 9$实际是要求:请往前找,直到找到一个不是9的数字,处理它。
例如的$0.\dot{9}$的$b_p$实际是对应首位的0,因此有$0.\dot{9}=1$)。


\textbf{【实数】}

在所有的无尽小数中,把每两个彼此等同的无尽小数视为同一个数,这样就得到了实数。
于是,每一个实数都具有唯一的规范小数表示。规范表示为$+a_0.a_1a_2\cdots$的实数被称为非负实数,其中规范表示为$+0.00\cdots$的实数记为$0$。
规范表示为$-b_0.b_1b_2\cdots$的实数被称为负实数。

\textbf{【相反数】}

两个非$0$实数,如果它们的规范小数表示的各位数字分别相同,但符号正好相反,那么我们就说这两个实数互为相反数。
$0$的相反数就规定为$0$自己。实数$x$的相反数通常记为$-x$。


\textbf{【实数的顺序】}

我们陈述比较两实数大小的规则如下:

\begin{enumerate}
  \item 两实数都是非负实数。对于规范表示的两个非负实数  
$a = a_0.a_1a_2\cdots a_n\cdots$ 和 $b = b_0.b_1b_2\cdots b_n\cdots$,我们逐位比较它们的各位数字。如果  
$a_0 = b_0, \cdots, a_{p-1} = b_{p-1}, a_p > b_p$,  
那么我们就说$a$大于$b$。
  \item 两实数都是负实数。对于规范表示的两个负实数  
$-c = -c_0.c_1c_2\cdots c_n\cdots$ 和 $-d = -d_0.d_1d_2\cdots d_n\cdots$,如果  
$c_0 = d_0, \cdots, c_{q-1} = d_{q-1}, c_q < d_q$,  
那么我们就说$-c$大于$-d$;
  \item 两实数之一是非负实数,另一个是负实数。对这情形,我们规定任何非负实数大于任何负实数。
\end{enumerate}



如果实数$x$大于实数$y$,那么我们就说实数$y$小于实数$x$。
如果两实数有相同的规范小数表示,那么我们就说这两实数相等。

用上述方式,我们在实数中定义了大于“$>$”、小于“$<$”和等于“$=$”等关系。
这样定义的顺序关系具有“三歧性”和“传递性”。


\textbf{【三歧性】}

对任意两个实数$a$和$b$,必有以下三种情形之一出现,而且也只有其中之一出现:

\begin{center}
$a>b$,$a=b$或者$a<b$
\end{center}

\textbf{【传递性】}

如果$a>b$,$b>c$,那么$a>c$。

我们约定:用记号$a\geq b$表示“$a>b$或者$a=b$”,
用记号$a\leq b$表示“$a<b$或者$a=b$”。


\textbf{【有尽小数在实数系中处处稠密】}

下面的定理指出,在任意两个不相等的实数之间,总能插进一个有尽小数。
这一重要结论说明了有尽小数在实数系中的稠密性。

\textbf{定理}:设$a$和$b$是实数,$a<b$,则存在有尽小数$c$,满足$a<c<b$。

\textbf{证明}:如果$a<0<b$,那么$c=0$就合乎要求。
因此只需考察$0\leq a<b$或者$a<b\leq0$的情形。
我们仅对$0\leq a<b$的情形证明,另一情形留给读者练习。

设$a$和$b$的规范小数表示为:
$$a=a_0.a_1a_2\cdots \quad \text{和} \quad b=b_0.b_1b_2\cdots$$
因为$a<b$,所以存在$p\in\mathbb{Z}^+$(正整数集),使得:
$$a_1=b_1,\cdots,a_{p-1}=b_{p-1},a_p<b_p$$
又因为$a_0.a_1a_2\cdots$是规范小数(不含无穷多个9的后缀),所以存在$q>p$,使得$a_q<9$。

我们取有尽小数:
$$c=a_0.a_1\cdots a_p \cdots  a_{q-1}(a_q+1)000\cdots$$
于是,$c$是有尽小数,它满足$$a<c<b$$。

\textbf{【实数的绝对值】}

实数$x$的绝对值$|x|$定义如下:
$$|x|=\begin{cases}
x, & \text{若}x\text{是非负实数}, \\
-x, & \text{若}x\text{是负实数}.
\end{cases}$$




\ifx\allfiles\undefined %若没定义主文件
\end{document}% 就打印这个结尾
\fi