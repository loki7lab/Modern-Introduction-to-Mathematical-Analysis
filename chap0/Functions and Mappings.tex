\ifx\allfiles\undefined% 如果这个没定义主文件
    \documentclass[12pt,a4paper,oneside,utf8]{ctexbook}
    \def\basepath{../}%basepath =  '../config'
    % -------------------------- 1. 路径与外部配置 --------------------------
% 定义基础路径变量(未提前定义则默认当前目录)
\ifx\basepath\undefined
    \newcommand{\basepath}{./}
\fi
% 引入外部宏包配置文件(按基础路径拼接)
% config/package.tex 
% 列出模板加载的包清单


% 基础数学和工具
\usepackage{amsmath} 
\usepackage{amsthm} 
\usepackage{amssymb}
\usepackage{mathrsfs} 
\usepackage{esint} 
\usepackage{yhmath} 
\usepackage{extarrows}
\usepackage{indentfirst}

% 颜色 (XColor 必须在 hyperref 之前)
\usepackage[dvipsnames, svgnames]{xcolor}
\usepackage[svgnames]{xcolor}
\usepackage{ulem} % 替代 soul



% 字体设置(fontspec 应该在 CJK 字体设置之前)
\usepackage{fontspec} 
\setCJKmainfont{Songti SC} 
\setmonofont{Courier New}
\setmainfont{Times New Roman}


% 布局和样式
\usepackage{geometry}
\usepackage{graphicx} 
\usepackage{fancyhdr} 
\usepackage{enumitem}
\usepackage[strict]{changepage} 
\usepackage{framed} 
\usepackage{listings}
% listings 依赖 xcolor,顺序没问题

\usepackage{setspace}  
\usepackage{microtype} 
\usepackage{titlesec}  

% 目录定制 (tocloft 必须在 hyperref 之前)
\usepackage{tocloft} 

% 超链接 (hyperref 必须是最后一个主要的宏包)
\usepackage{hyperref}

% 设置图片文件根路径(基础路径下的figure文件夹)
\graphicspath{{\basepath figure/}}


% -------------------------- 2. 页面尺寸与边距 --------------------------
\geometry{ 
    papersize={210mm,285mm}, % 页面尺寸(接近A4,略短)
    left=22mm, right=22mm,   % 左右边距
    top=28mm, bottom=28mm,   % 上下边距
    headheight=14pt, headsep=20pt % 页眉高度与间距
}


% -------------------------- 3. 章节标题样式 --------------------------
% 部分(Part)标题:新页+居中+特大号加粗
\titleformat{\part}{\newpage\centering\Huge\bfseries}{}{0em}{}
\titlespacing*{\part}{0pt}{0ex}{3.0ex} % 部分标题间距

% 章节(Chapter)标题:居中+大号加粗+带章节号
\titleformat{\chapter}{\centering\Large\bfseries}{\thechapter.}{1em}{}
\titlespacing*{\chapter}{0pt}{3.0ex}{2.0ex} % 章节标题间距

% 小节(Section)标题:大号加粗+带§符号
\titleformat{\section}{\centering\large\bfseries}{§\arabic{section}}{0.8em}{}
\titlespacing*{\section}{0pt}{2.5ex}{1.5ex} % 小节标题间距

% 自定义未编号章节(Explanation):封装chapter*,复用其样式
\newcommand{\explanation}[1]{\chapter*{#1}}



% -------------------------- 4. 段落与列表格式 --------------------------
\parindent=2em % 段落首行缩进2字符

% 有序/无序/描述列表统一格式:调整间距
\setenumerate[1]{itemsep=5pt, partopsep=0pt, parsep=\parskip, topsep=5pt}
\setitemize[1]{itemsep=5pt, partopsep=0pt, parsep=\parskip, topsep=5pt}
\setdescription{itemsep=5pt, partopsep=0pt, parsep=\parskip, topsep=5pt}

% 重新定义 \hl 命令
\definecolor{lightpink}{rgb}{1.0, 0.71, 0.76}
\newcommand{\hl}[1]{%
  \bgroup
  \markoverwith{\textcolor{lightpink}{\rule[-.5ex]{2pt}{2.5ex}}}%
  \ULon{#1}%
}


% -------------------------- 5. 页眉页脚配置 --------------------------
% -------------------------- 5. 页眉页脚配置 --------------------------
\setlength{\headheight}{15pt} % 解决 headheight too small 警告
\pagestyle{fancy} % 使用fancy样式
\fancyhf{} % 清空默认内容
\fancyhead[L]{\leftmark} % 左页眉:当前章节标题
\fancyhead[R]{\thepage} % 右页眉:页码
\renewcommand{\headrulewidth}{0.4pt} % 页眉下边框线宽
\renewcommand{\footrulewidth}{0pt} % 隐藏页脚下边框


% -------------------------- 6. 代码块样式(分语言定义) --------------------------
% 定义Mathematica代码样式(避免全局覆盖)
\lstdefinestyle{MathematicaStyle}{
    language=Mathematica, basicstyle=\tt, breaklines=true,
    keywordstyle=\bfseries\color{NavyBlue}, emphstyle=\bfseries\color{Rhodamine},
    commentstyle=\itshape\color{black!50!white}, stringstyle=\bfseries\color{PineGreen!90!black},
    columns=flexible, numbers=left, numberstyle=\footnotesize, frame=tb, breakatwhitespace=false
}

% 定义TeX代码样式(避免全局覆盖)
\lstdefinestyle{TeXStyle}{
    language=TeX, basicstyle=\ttfamily, breaklines=true,
    keywordstyle=\bfseries\color{NavyBlue}, emphstyle=\bfseries\color{Rhodamine},
    commentstyle=\itshape\color{black!50!white}, stringstyle=\bfseries\color{PineGreen!90!black},
    columns=flexible, numbers=left, numberstyle=\footnotesize, frame=tb, breakatwhitespace=false
}


% 可选:设置全局默认代码样式(按需启用)
% \lstset{style=MathematicaStyle}% 传进去后有 ../config/package.tex
    \begin{document}
\else% 如果定义了
\fi



\section{函数与映射}


人们常说:“变量$y$随着变量$\alpha$的变化而变化”或者“变量$y$是变量$x$的函数”。这些说法的确切含义是什么呢?这就是说:变量$x$所取的任何一个确定的值,决定了变量$y$的唯一确定的值。或者说:对变量$x$的任何一个值,有变量$y$的唯一确定的值与之对应。采用集合论的术语对这些说法做进一步的概括,就得到映射的概念。

设$D$和$E$都是集合。我们把$D$的元素与$E$的元素之间的对应关系$f$叫作一个映射,如果按照这对应关系,对集合$D$中的任何一个元素$\xi$,有集合$E$中唯一的一个元素$\eta$与之对应。$f$是从$D$到$E$的一个映射这件事,通常记为:


\[ f: D \to E \]

按照对应关系$f$,由$D$中的元素$\xi$所决定的$E$中的唯一元素$\eta$记为$f(\xi)$。有时候,我们用记号$\xi \mapsto \eta$表示元素之间的对应。例如,设$D = \mathbb{R}$,$E = \mathbb{R}$,而映射$f:D \to E$定义为$f(x) = x^2$,则这映射规定了元素之间这样的对应关系:


\[ f: x \mapsto x^2 \]

设$f:D \to E$是一个映射,$A \subset D$,$B \subset E$。我们把集合


\[ f(A) = \{ f(x) \mid x \in A \} (\subset E) \]
叫作集合$A$经过映射$f$的像集,并把集合


\[ f^{-1}(B) = \{ x \mid f(x) \in B \} (\subset D) \]
叫作集合$B$关于映射$f$的原像集。

如果$D \subset \mathbb{R}$,$E = \mathbb{R}$,那么从$D$到$E$的映射就是通常的一元函数。但映射的概念远比这广泛。在以后的学习中,将会遇到更广泛的映射的例子。但在开始的时候,我们主要关心的是函数。

\subsection*{例1:圆的面积函数}
圆的面积$S$是半径$r$的函数:


\[ S = \pi r^2 \]
在这里,$D = \mathbb{R}^+$(正实数集),$E = \mathbb{R}$(实数集),对应关系由一个代数运算式来表示。


\subsection*{例2:自由落体路程函数}
自由落体经过的路程$s$是时间$t$的函数:\[ s = \frac{1}{2}gt^2 \]这里的函数关系也能用一个代数式来表示。

\subsection*{例3:赫维赛德(Heaviside)函数}
有一些函数关系具有“分段”的表达形式,例如在技术科学中有重要应用的赫维赛德(Heaviside)函数
(又称单位阶跃函数)可以表示为:\[ H(t) = \begin{cases} 
-1, & t < 0, \\
0, & t = 0, \\
1, & t > 0 
\end{cases} \]

\subsection*{例4:狄利克雷(Dirichlet)函数}
狄利克雷(Dirichlet)函数定义如下:\[ D(t) = \begin{cases} 
1, & t \text{ 是有理数}, \\
0, & t \text{ 是无理数} 
\end{cases} \]或用集合符号简化表示为:\[ D(t) = \begin{cases}1, & t \in \mathbb{Q}, \\0, & t \notin \mathbb{Q}\end{cases} \]

\subsection*{例5:气压随时间变化的函数}
在自动记录气压计中,有一个匀速转动的圆柱形记录鼓。印有坐标方格的记录纸就裹在这个鼓上。记录鼓每24小时转动一周。气压计指针的端点装有一支墨水笔,笔尖接触着记录纸。这样,经过24小时之后,取下的记录纸上就描画了一条曲线。这条曲线表示气压$p$随时间$t$变化的函数关系。

\subsection*{例6:自然数编号实数序列}
设$D = \mathbb{N}$(自然数集),$E = \mathbb{R}$(实数集)。一个映射\[ f: \mathbb{N} \to \mathbb{R} \]意味着用自然数编号的一串实数:\[ x_1 = f(1),\ x_2 = f(2),\ \cdots,\ x_n = f(n),\ \cdots \]这样的一个映射,或者说这样的以自然数编号的一串实数$\{x_n\}$,被称为实数序列。

现在补充复合函数的定义。设$f: D \to E$是一个映射,$g: G \to H$也是一个映射。如果$f(D) \subset G$,那么从$\xi \in D$开始,相继经过$f$和$g$的作用,就得到$g(f(\xi))$。这样的对应关系\[ \xi \mapsto g(f(\xi)) \]也是一个映射。我们把这个映射称为$g$与$f$的复合,记为$g \circ f$。简言之,映射$g$与映射$f$的复合定义为\[ g \circ f: D \to H,\quad \xi \mapsto g(f(\xi)) \]


\subsection*{例7:幂函数的复合}
设$f:\mathbb{R} \to \mathbb{R}$定义为$f(x)=x^n$,则有:\[ f \circ f(x) = f(f(x)) = (x^n)^n = x^{n^2} \]

\subsection*{例8:函数复合的顺序差异}
考察函数$f(x)=x^2$和$g(x)=\sin x$,我们有:\[ g \circ f(x) = \sin(x^2),\quad f \circ g(x) = (\sin x)^2 = \sin^2 x \]

一般说来,对于映射$f$和$g$,两种顺序的复合映射$g \circ f$和$f \circ g$不一定都有定义,即使有定义也不一定相同。让我们再看两个例子。

\subsection*{例9:复合映射的定义域限制}
考察函数$f(x)=\sqrt{1-x}$和$g(x)=x^2+10$。我们看到$g \circ f$对$x \leq 1$有定义而对$f \circ g$无定义。

\subsection*{例10:复合映射的结果差异}
考察函数$f(x)=x^2$和$g(x)=x+2$。这两者均在$\mathbb{R}$上有定义,
因而\[ g \circ f(x) = (x^2) + 2 = x^2 + 2 \]\[ f \circ g(x) = (x+2)^2 = x^2 + 4x + 4 \]



\ifx\allfiles\undefined %若没定义主文件
\end{document}% 就打印这个结尾
\fi