\ifx\allfiles\undefined% 如果这个没定义主文件
    \documentclass[12pt,a4paper,oneside,utf8]{ctexbook}
    \def\basepath{../}%basepath =  '../config'
    % -------------------------- 1. 路径与外部配置 --------------------------
% 定义基础路径变量(未提前定义则默认当前目录)
\ifx\basepath\undefined
    \newcommand{\basepath}{./}
\fi
% 引入外部宏包配置文件(按基础路径拼接)
% config/package.tex 
% 列出模板加载的包清单


% 基础数学和工具
\usepackage{amsmath} 
\usepackage{amsthm} 
\usepackage{amssymb}
\usepackage{mathrsfs} 
\usepackage{esint} 
\usepackage{yhmath} 
\usepackage{extarrows}
\usepackage{indentfirst}

% 颜色 (XColor 必须在 hyperref 之前)
\usepackage[dvipsnames, svgnames]{xcolor}
\usepackage[svgnames]{xcolor}
\usepackage{ulem} % 替代 soul



% 字体设置(fontspec 应该在 CJK 字体设置之前)
\usepackage{fontspec} 
\setCJKmainfont{Songti SC} 
\setmonofont{Courier New}
\setmainfont{Times New Roman}


% 布局和样式
\usepackage{geometry}
\usepackage{graphicx} 
\usepackage{fancyhdr} 
\usepackage{enumitem}
\usepackage[strict]{changepage} 
\usepackage{framed} 
\usepackage{listings}
% listings 依赖 xcolor,顺序没问题

\usepackage{setspace}  
\usepackage{microtype} 
\usepackage{titlesec}  

% 目录定制 (tocloft 必须在 hyperref 之前)
\usepackage{tocloft} 

% 超链接 (hyperref 必须是最后一个主要的宏包)
\usepackage{hyperref}

% 设置图片文件根路径(基础路径下的figure文件夹)
\graphicspath{{\basepath figure/}}


% -------------------------- 2. 页面尺寸与边距 --------------------------
\geometry{ 
    papersize={210mm,285mm}, % 页面尺寸(接近A4,略短)
    left=22mm, right=22mm,   % 左右边距
    top=28mm, bottom=28mm,   % 上下边距
    headheight=14pt, headsep=20pt % 页眉高度与间距
}


% -------------------------- 3. 章节标题样式 --------------------------
% 部分(Part)标题:新页+居中+特大号加粗
\titleformat{\part}{\newpage\centering\Huge\bfseries}{}{0em}{}
\titlespacing*{\part}{0pt}{0ex}{3.0ex} % 部分标题间距

% 章节(Chapter)标题:居中+大号加粗+带章节号
\titleformat{\chapter}{\centering\Large\bfseries}{\thechapter.}{1em}{}
\titlespacing*{\chapter}{0pt}{3.0ex}{2.0ex} % 章节标题间距

% 小节(Section)标题:大号加粗+带§符号
\titleformat{\section}{\centering\large\bfseries}{§\arabic{section}}{0.8em}{}
\titlespacing*{\section}{0pt}{2.5ex}{1.5ex} % 小节标题间距

% 自定义未编号章节(Explanation):封装chapter*,复用其样式
\newcommand{\explanation}[1]{\chapter*{#1}}



% -------------------------- 4. 段落与列表格式 --------------------------
\parindent=2em % 段落首行缩进2字符

% 有序/无序/描述列表统一格式:调整间距
\setenumerate[1]{itemsep=5pt, partopsep=0pt, parsep=\parskip, topsep=5pt}
\setitemize[1]{itemsep=5pt, partopsep=0pt, parsep=\parskip, topsep=5pt}
\setdescription{itemsep=5pt, partopsep=0pt, parsep=\parskip, topsep=5pt}

% 重新定义 \hl 命令
\definecolor{lightpink}{rgb}{1.0, 0.71, 0.76}
\newcommand{\hl}[1]{%
  \bgroup
  \markoverwith{\textcolor{lightpink}{\rule[-.5ex]{2pt}{2.5ex}}}%
  \ULon{#1}%
}


% -------------------------- 5. 页眉页脚配置 --------------------------
% -------------------------- 5. 页眉页脚配置 --------------------------
\setlength{\headheight}{15pt} % 解决 headheight too small 警告
\pagestyle{fancy} % 使用fancy样式
\fancyhf{} % 清空默认内容
\fancyhead[L]{\leftmark} % 左页眉:当前章节标题
\fancyhead[R]{\thepage} % 右页眉:页码
\renewcommand{\headrulewidth}{0.4pt} % 页眉下边框线宽
\renewcommand{\footrulewidth}{0pt} % 隐藏页脚下边框


% -------------------------- 6. 代码块样式(分语言定义) --------------------------
% 定义Mathematica代码样式(避免全局覆盖)
\lstdefinestyle{MathematicaStyle}{
    language=Mathematica, basicstyle=\tt, breaklines=true,
    keywordstyle=\bfseries\color{NavyBlue}, emphstyle=\bfseries\color{Rhodamine},
    commentstyle=\itshape\color{black!50!white}, stringstyle=\bfseries\color{PineGreen!90!black},
    columns=flexible, numbers=left, numberstyle=\footnotesize, frame=tb, breakatwhitespace=false
}

% 定义TeX代码样式(避免全局覆盖)
\lstdefinestyle{TeXStyle}{
    language=TeX, basicstyle=\ttfamily, breaklines=true,
    keywordstyle=\bfseries\color{NavyBlue}, emphstyle=\bfseries\color{Rhodamine},
    commentstyle=\itshape\color{black!50!white}, stringstyle=\bfseries\color{PineGreen!90!black},
    columns=flexible, numbers=left, numberstyle=\footnotesize, frame=tb, breakatwhitespace=false
}


% 可选:设置全局默认代码样式(按需启用)
% \lstset{style=MathematicaStyle}% 传进去后有 ../config/package.tex
    \begin{document}
\else% 如果定义了
\fi

\section{面积、路程与功的计算}


我们已经会求直线图形和圆的面积。为了计算更一般的曲线图形的面积,需要寻求更有效的方法。

先来看一个具体的例子。设有这样一个曲线图形,它由曲线 $y=x^p$,
$OX$ 轴和直线 $x=b$ 围成,我们来求它的面积(图 0-1)。

\begin{figure}[h]
    \centering
    % The \graphicspath in config.tex handles the folder location
    \includegraphics[width=0.6\textwidth]{0_1.jpg}
    \caption{非负的曲面面积计算}
    \label{fig:0_1}
\end{figure}


我们把 $OX$ 轴上的闭区间 $[0, b]$ 分成 $n$ 等分,
其中第 $k$ 个等分是 
\[\left[ \dfrac{(k-1)b}{n}, \dfrac{kb}{n} \right].\]

相应地把上述曲线图形分成 $n$ 个等宽的条形:
\[
\frac{k-1}{n}b \leq x \leq \frac{k}{n}b, \quad 0 \leq y \leq x^p, \quad k=1,2,\cdots,n.
\]


每一条形的面积 $S_k$ 介于二矩形条的面积之间:
\[
\left( \frac{(k-1)b}{n} \right)^p \cdot \frac{b}{n}
 \leq S_k 
 \leq \left( \frac{kb}{n} \right)^2  \cdot \frac{b}{n}.
\]

因而所求的曲线图形的面积 $S$ 应该介于以下两个和数之间:
\[
\sum_{k=1}^n \left( \dfrac{(k-1)}{n}b \right)^p \cdot
 \dfrac{b}{n} \leq S \leq 
 \sum_{k=1}^n \left( \dfrac{k}{n}b \right)^p \cdot \dfrac{b}{n}.
\]

我们可以把矩形条面积之和当作曲线图形面积 $S$ 的近似值。所分成的矩形条越细,
这样的近似值的精确度就越高。事实上,分别处理两个和数,我们有
\begin{align*}
\sum_{k=1}^{n} \left( \dfrac{k}{n}b \right)^p \cdot \dfrac{b}{n} 
&= \dfrac{b^{p+1}}{n^{p+1}} \sum_{k=1}^n k^p \\
&= \dfrac{b^{p+1}}{n^{p+1}} \left( \dfrac{1}{p+1}n^{p+1} + c_1 n^p + \cdots + c_p n \right)\\
&= b^{p+1} \left( \dfrac{1}{p+1} + \dfrac{c_1}{n} + \cdots + \dfrac{c_p}{n^p}
\right).
\end{align*}


\begin{align*}
\sum_{k=1}^{n} \left( \dfrac{k-1}{n}b \right)^p \cdot \dfrac{b}{n} 
&= \sum_{k=1}^{n}\left(\dfrac{k}{n}b \right)^p \cdot \dfrac{b}{n}-\dfrac{b^{p+1}}{n}\\
&= b^{p+1}\left(\dfrac{1}{p+1}+\dfrac{c_1 - 1}{n}+\cdots +\dfrac{c_p}{n^p}\right).
\end{align*}

当 $n$ 无限增大时,上面两个和数趋于共同的极限值 $\dfrac{b^{p+1}}{p+1}$,
这共同的极限值应该看作所求的面积 $S$。这样,我们求得
\[
S = \dfrac{b^{p+1}}{p+1}.
\]

再来看一般的情形。设函数 $y=f(x)$ 在闭区间 $[a,b]$ 上有定义并且非负
(即只取大于或等于 $0$ 的值)。
曲线 $y=f(x)$ 与直线 $x=a,\ x=b,\ y=0$ 围成一个图形,我们来求这个曲线图形的面积 $S$。
为此,用一串分点
\[
a = x_0 < x_1 < \cdots < x_n = b
\]
把闭区间 $[a,b]$ 分成 $n$ 段,相应地把上述曲线图形分成 $n$ 个条形,
其中第 $j$ 个条形为 $x_{j-1} \leq x \leq x_j,\ 0 \leq y \leq f(x)$。

在闭区间 $[x_{j-1}, x_j]$ 上任取一点$\xi_j$,我们把高为 $f(\xi_j)$,
底长为 $\Delta x_j = x_j - x_{j-1}$ 的矩形条的面积,
当作曲线图形的第 $j$ 个条形的面积的近似值。这样得到曲线图形面积的近似值
\[
\sum_{j=1}^n f(\xi_j) \Delta x_j.
\]

以后将证明,对于相当普遍的函数 $f$,当分割的条形越来越窄时,上述和式有确定的极限。
这极限应当视为所求曲线图形的面积。

我们还可以讨论更一般的情形。设 $y=f(x)$ 是在闭区间 $[a,b]$ 上有定义的函数(不一定非负),
考察由直线 $x=a,\ x=b,\ y=0$ 与曲线 $y=f(x)$ 所围图形的面积。

\begin{figure}[h]
    \centering
    % The \graphicspath in config.tex handles the folder location
    \includegraphics[width=0.6\textwidth]{0_2.jpg}
    \caption{一般性的曲面面积计算}
    \label{fig:0_2}
\end{figure}

我们约定,对于函数 $f$ 取负值的部分,曲线 $y=f(x)$ 与 $OX$ 轴所夹的面积为负值(图 0-2)。
这样,我们仍能把所述图形的面积的近似值表示为



\[
\sum_{j=1}^n f(\xi_j) \Delta x_j.
\]


对于相当普遍的函数 $f$,当分割的条形越来越窄时,上述和式有确定的极限。这极限应当视为所述曲线图形的面积的代数值。

再来看一些取自物理学的例子。

设物体做变速直线运动,其速度 $v$ 是时间 $t$ 的函数 $v=f(t)$。
我们来计算这个物体从时刻 $a$ 到时刻 $b$ 经过的路程。为此,用一串分点
\[
a = t_0 < t_1 < \cdots < t_n = b
\]
把这段时间分成 $n$ 小段。在第 $j$ 段时间中物体通过的路程可以认为近似等于
\[
f(\tau_j) \Delta t_j,
\]
这里 $\tau_j$ 是 $[t_{j-1}, t_j]$ 中的一个时刻(作为这个小段的代表速度),
$\Delta t_j = t_j - t_{j-1}$。于是,从时刻 $a$ 到时刻 $b$ 物体通过的路程近似等于
\[
\sum_{j=1}^n f(\tau_j) \Delta t_j.
\]
当所分割的时间间隔越来越短时,上述和式的极限值即为物体从时刻 $a$ 到时刻 $b$ 通过的路程。

另一取自物理学的问题是求变力所做的功。设物体 $m$ 受到一个沿 $OX$ 轴方向的力 $F$ 的作用,
它沿这个轴从 $a$ 点运动到 $b$ 点。如果力 $F$ 随着 $x$ 而改变,即 $$F = f(x)$$,
我们来求 $F$ 对这物体 $m$ 所做的功。为此,在 $a$ 和 $b$ 之间插入一串分点
\[
a = x_0 < x_1 < \cdots < x_n = b.
\]
设 $\xi_j$ 是 $[x_{j-1}, x_j]$ 上任意一点,而 $\Delta x_j = x_j - x_{j-1}$。
在路程 $[x_{j-1}, x_j]$ 上把力 $F=f(x)$ 近似地看成常力 $f(\xi_j)$,
则在这段上力 $F$ 所做的功近似地等于
\[
f(\xi_j) \Delta x_j.
\]
变力 $F=f(x)$ 在整段路程 $[a,b]$ 上所做的功近似地等于
\[
\sum_{j=1}^n f(\xi_j) \Delta x_j.
\]
当所分割的路程间隔越来越小时,上述和式的极限值即为变力 $F=f(x)$ 所做的功 $W$。

在上面列举的例子中,来源不同的几个问题都可以用类似的方法讨论。还可以举出更多的例子,所涉及的问题归结为如下形式的和数的极限
\[
\sum_{j=1}^n f(\xi_j) \Delta x_j.
\]
当分割无限加细的时候,上述和数的极限值称为函数 $f(x)$ 的积分,记为
\[
\int_a^b f(x) \, dx,
\]
其中 $a,b$ 表示和数展布的区间,积分号 $\int$(拉长了的 $S$)表示求和求极限的过程,
而 $f(x)\,dx$ 表示求和各项的形状。
如果把 $\Delta x_1, \Delta x_2,\dots, \Delta x_n$ 中最大的一个记为
\[
\max \Delta x_j,
\]
那么我们就可以写
\[
\int_a^b f(x) \, dx = \lim_{\max \Delta x_j \to 0} \sum_{j=1}^n f(\xi_j) \Delta x_j.
\]

早在公元前 3 世纪,古希腊时代的著名学者阿基米德(Archimedes)就已经会计算曲线图形
\[
0 \leq x \leq b, \quad 0 \leq y \leq x^2
\]
的面积。但他的方法(所谓“穷竭法”)陈述起来并不那么简单清楚,所以在很长的一段时间里没有被人们普遍接受。直到两千年以后,牛顿(Newton)和莱布尼茨(Leibniz)创立了微积分学,特别是把积分的计算与微分联系起来,人们才有了统一地解决多种多样的问题的简单而有效的工具。



\ifx\allfiles\undefined %若没定义主文件
\end{document}% 就打印这个结尾
\fi