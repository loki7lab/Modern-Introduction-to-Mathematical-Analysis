\ifx\allfiles\undefined% 如果这个没定义主文件
    \documentclass[12pt,a4paper,oneside,utf8]{ctexbook}
    \def\basepath{../}%basepath =  '../config'
    % -------------------------- 1. 路径与外部配置 --------------------------
% 定义基础路径变量(未提前定义则默认当前目录)
\ifx\basepath\undefined
    \newcommand{\basepath}{./}
\fi
% 引入外部宏包配置文件(按基础路径拼接)
% config/package.tex 
% 列出模板加载的包清单


% 基础数学和工具
\usepackage{amsmath} 
\usepackage{amsthm} 
\usepackage{amssymb}
\usepackage{mathrsfs} 
\usepackage{esint} 
\usepackage{yhmath} 
\usepackage{extarrows}
\usepackage{indentfirst}

% 颜色 (XColor 必须在 hyperref 之前)
\usepackage[dvipsnames, svgnames]{xcolor}
\usepackage[svgnames]{xcolor}
\usepackage{ulem} % 替代 soul



% 字体设置(fontspec 应该在 CJK 字体设置之前)
\usepackage{fontspec} 
\setCJKmainfont{Songti SC} 
\setmonofont{Courier New}
\setmainfont{Times New Roman}


% 布局和样式
\usepackage{geometry}
\usepackage{graphicx} 
\usepackage{fancyhdr} 
\usepackage{enumitem}
\usepackage[strict]{changepage} 
\usepackage{framed} 
\usepackage{listings}
% listings 依赖 xcolor,顺序没问题

\usepackage{setspace}  
\usepackage{microtype} 
\usepackage{titlesec}  

% 目录定制 (tocloft 必须在 hyperref 之前)
\usepackage{tocloft} 

% 超链接 (hyperref 必须是最后一个主要的宏包)
\usepackage{hyperref}

% 设置图片文件根路径(基础路径下的figure文件夹)
\graphicspath{{\basepath figure/}}


% -------------------------- 2. 页面尺寸与边距 --------------------------
\geometry{ 
    papersize={210mm,285mm}, % 页面尺寸(接近A4,略短)
    left=22mm, right=22mm,   % 左右边距
    top=28mm, bottom=28mm,   % 上下边距
    headheight=14pt, headsep=20pt % 页眉高度与间距
}


% -------------------------- 3. 章节标题样式 --------------------------
% 部分(Part)标题:新页+居中+特大号加粗
\titleformat{\part}{\newpage\centering\Huge\bfseries}{}{0em}{}
\titlespacing*{\part}{0pt}{0ex}{3.0ex} % 部分标题间距

% 章节(Chapter)标题:居中+大号加粗+带章节号
\titleformat{\chapter}{\centering\Large\bfseries}{\thechapter.}{1em}{}
\titlespacing*{\chapter}{0pt}{3.0ex}{2.0ex} % 章节标题间距

% 小节(Section)标题:大号加粗+带§符号
\titleformat{\section}{\centering\large\bfseries}{§\arabic{section}}{0.8em}{}
\titlespacing*{\section}{0pt}{2.5ex}{1.5ex} % 小节标题间距

% 自定义未编号章节(Explanation):封装chapter*,复用其样式
\newcommand{\explanation}[1]{\chapter*{#1}}



% -------------------------- 4. 段落与列表格式 --------------------------
\parindent=2em % 段落首行缩进2字符

% 有序/无序/描述列表统一格式:调整间距
\setenumerate[1]{itemsep=5pt, partopsep=0pt, parsep=\parskip, topsep=5pt}
\setitemize[1]{itemsep=5pt, partopsep=0pt, parsep=\parskip, topsep=5pt}
\setdescription{itemsep=5pt, partopsep=0pt, parsep=\parskip, topsep=5pt}

% 重新定义 \hl 命令
\definecolor{lightpink}{rgb}{1.0, 0.71, 0.76}
\newcommand{\hl}[1]{%
  \bgroup
  \markoverwith{\textcolor{lightpink}{\rule[-.5ex]{2pt}{2.5ex}}}%
  \ULon{#1}%
}


% -------------------------- 5. 页眉页脚配置 --------------------------
% -------------------------- 5. 页眉页脚配置 --------------------------
\setlength{\headheight}{15pt} % 解决 headheight too small 警告
\pagestyle{fancy} % 使用fancy样式
\fancyhf{} % 清空默认内容
\fancyhead[L]{\leftmark} % 左页眉:当前章节标题
\fancyhead[R]{\thepage} % 右页眉:页码
\renewcommand{\headrulewidth}{0.4pt} % 页眉下边框线宽
\renewcommand{\footrulewidth}{0pt} % 隐藏页脚下边框


% -------------------------- 6. 代码块样式(分语言定义) --------------------------
% 定义Mathematica代码样式(避免全局覆盖)
\lstdefinestyle{MathematicaStyle}{
    language=Mathematica, basicstyle=\tt, breaklines=true,
    keywordstyle=\bfseries\color{NavyBlue}, emphstyle=\bfseries\color{Rhodamine},
    commentstyle=\itshape\color{black!50!white}, stringstyle=\bfseries\color{PineGreen!90!black},
    columns=flexible, numbers=left, numberstyle=\footnotesize, frame=tb, breakatwhitespace=false
}

% 定义TeX代码样式(避免全局覆盖)
\lstdefinestyle{TeXStyle}{
    language=TeX, basicstyle=\ttfamily, breaklines=true,
    keywordstyle=\bfseries\color{NavyBlue}, emphstyle=\bfseries\color{Rhodamine},
    commentstyle=\itshape\color{black!50!white}, stringstyle=\bfseries\color{PineGreen!90!black},
    columns=flexible, numbers=left, numberstyle=\footnotesize, frame=tb, breakatwhitespace=false
}


% 可选:设置全局默认代码样式(按需启用)
% \lstset{style=MathematicaStyle}% 传进去后有 ../config/package.tex
    \begin{document}
\else% 如果定义了
\fi



\section{连加符号与连乘符号}


在数学中,常遇到一连串的数相加或者一连串的数相乘,例如 $1+2+\cdots+n$ 
或者 $m(m-1)\cdots(m-k+1)$ 等。为简便起见,人们引入连加符号 $\sum$ 与连乘符号 $\prod$:



\[
\sum_{i=1}^n x_i = x_1 + x_2 + \cdots + x_n
\]

\[
\prod_{i=1}^n x_i = x_1 x_2 \cdots x_n
\]

这里的指标 $i$ 仅仅用以表示求和或求乘积的范围,把 $i$ 换成别的符号 $j,k$ 等,
也仍然表示同一和或同一乘积,例如:



\[
\sum_{i=1}^n x_i = x_1 + x_2 + \cdots + x_n = \sum_{j=1}^n x_j
\]

\[
\prod_{i=1}^n x_i = x_1 x_2 \cdots x_n = \prod_{k=1}^n x_k
\]人们通常把这样的指标称为“哑指标”。


我们举几个例子说明连加符号与连乘符号的应用。

\textbf{例1} 阶乘 $n!$ 的定义可以写成:


\[
n! = \prod_{j=1}^n j
\]

\textbf{例2} 二项式定理可以表示为:


\[
(a+b)^n = \sum_{j=0}^n \binom{n}{j} a^{j} b^{n-j} = \sum_{k=0}^n \binom{n}{k} a^{n-k} b^k
\]
这里组合数的定义为:


\[
\binom{n}{k} = \frac{n(n-1)\cdots(n-k+1)}{k!}
\]


\textbf{例3} $\sum_{k=1}^n 1 = \underbrace{1+1+\cdots+1}_\text{n项} = n$

\textbf{例4} 我们来计算 $\sum_{k=1}^n [k^p - (k-1)^p]$,这个和式表示:


\[
(1^p - 0^p) + (2^p -1^p) + \cdots + (n^p - (n-1)^p)
\]
因而:


\[
\sum_{k=1}^n [k^p - (k-1)^p] = n^p
\]


数的运算满足交换律、结合律以及(乘法对加法的)分配律,据此得到以下运算法则:



\[
\sum_{i=1}^n (a_i + b_i) = (a_1 + b_1) + \cdots + (a_n + b_n) = \sum_{i=1}^n a_i + \sum_{i=1}^n b_i
\]


\[
\sum_{i=1}^n (\lambda c_i) = \lambda(c_1 + \cdots + c_n) = \lambda \sum_{i=1}^n c_i
\]

\textbf{例5} 我们有恒等式:


\[
k^2 - (k-1)^2 = 2k -1
\]


对于$k=1,2,\cdots,n$,将恒等式$k^2-(k-1)^2=2k-1$累加,得:  


\[
\sum_{k=1}^n [k^2-(k-1)^2] = 2\sum_{k=1}^n k - \sum_{k = 1}^{n}1
\]  

根据例4的结论,整理左边有
\[
n^2 = 2\sum_{k=1}^{n}k - n
\]

\[
\sum_{k = 1}^{n}k = \frac{1}{2}n^2 + \frac{1}{2}n = \frac{n(n+1)}{2}
\]

我们得到了熟悉的公式

\[
1+2+\cdots+n=\frac{n(n+1)}{2}
\]

\textbf{例6} 由恒等式\[k^3-(k-1)^3=3k^2-3k+1\]可得:  


\[
\sum_{k=1}^n [k^3-(k-1)^3] = 3\sum_{k=1}^n k^2 -3\sum_{k=1}^n k + \sum_{k=1}^n 1\]  

而根据例4整理左边有
\[
n^3 = 3\sum_{k = 1}^{n} k^2 -3(\frac{1}{2}n^2+\frac{1}{2}n)+n
\]

进一步整理上面的式子,有
\[
\sum_{k=1}^{n}k^2 = \frac{1}{3}n^3+\frac{1}{2}n^2+\frac{1}{6}n = (\frac{n(1+n)(2n+1)}{6})
\]


类似地,由恒等式

\[
k^4-(k-1)^4=4k^3-6k^2+4k-1
\]

可得

\[
\sum_{k=1}^{4}(k^4-(k-1)^4)=4\sum_{k=1}^{n}k^3-6\sum_{k=1}^{n}k^2+4\sum_{k=1}^{n}k-\sum_{k=1}^{n}1
\]

整理左边有
\[
n^4=4\sum_{k=1}^{n}k^3-6\sum_{k=1}^{n}k^2+4\sum_{k=1}^{n}k-\sum_{k=1}^{n}1
\]

也就是
\[
\sum_{k=1}^{n}k^3 = \frac{1}{4}n^4+\frac{1}{2}n^3+\frac{1}{4}n^2 = (\frac{n(n+1)}{2})^2
\]

采用类似的推理方式,利用数学归纳法可以证明以下结论:
$\sum_{k=1}^{n}k^p$可以表示$n$的$p+1$次多项式,其最高系数$\frac{1}{p+1}$,常数项为0,即

\[
\sum_{k=1}^{n}k^p = \frac{1}{p+1}n^{p+1}+c_1n^p+c_2n^{p-1}+\cdots+c_p n
\]

对于给定$p$,上面公式中的系数$c_1,\cdots ,c_p$当然都可以具体算出。我们这里不再做深入的讨论了。
\ifx\allfiles\undefined %若没定义主文件
\end{document}% 就打印这个结尾
\fi