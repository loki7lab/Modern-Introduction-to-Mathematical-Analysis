\ifx\allfiles\undefined% 如果这个没定义主文件
    \documentclass[12pt,a4paper,oneside,utf8]{ctexbook}
    \def\basepath{../}%basepath =  '../config'
    % -------------------------- 1. 路径与外部配置 --------------------------
% 定义基础路径变量(未提前定义则默认当前目录)
\ifx\basepath\undefined
    \newcommand{\basepath}{./}
\fi
% 引入外部宏包配置文件(按基础路径拼接)
% config/package.tex 
% 列出模板加载的包清单


% 基础数学和工具
\usepackage{amsmath} 
\usepackage{amsthm} 
\usepackage{amssymb}
\usepackage{mathrsfs} 
\usepackage{esint} 
\usepackage{yhmath} 
\usepackage{extarrows}
\usepackage{indentfirst}

% 颜色 (XColor 必须在 hyperref 之前)
\usepackage[dvipsnames, svgnames]{xcolor}
\usepackage[svgnames]{xcolor}
\usepackage{ulem} % 替代 soul



% 字体设置(fontspec 应该在 CJK 字体设置之前)
\usepackage{fontspec} 
\setCJKmainfont{Songti SC} 
\setmonofont{Courier New}
\setmainfont{Times New Roman}


% 布局和样式
\usepackage{geometry}
\usepackage{graphicx} 
\usepackage{fancyhdr} 
\usepackage{enumitem}
\usepackage[strict]{changepage} 
\usepackage{framed} 
\usepackage{listings}
% listings 依赖 xcolor,顺序没问题

\usepackage{setspace}  
\usepackage{microtype} 
\usepackage{titlesec}  

% 目录定制 (tocloft 必须在 hyperref 之前)
\usepackage{tocloft} 

% 超链接 (hyperref 必须是最后一个主要的宏包)
\usepackage{hyperref}

% 设置图片文件根路径(基础路径下的figure文件夹)
\graphicspath{{\basepath figure/}}


% -------------------------- 2. 页面尺寸与边距 --------------------------
\geometry{ 
    papersize={210mm,285mm}, % 页面尺寸(接近A4,略短)
    left=22mm, right=22mm,   % 左右边距
    top=28mm, bottom=28mm,   % 上下边距
    headheight=14pt, headsep=20pt % 页眉高度与间距
}


% -------------------------- 3. 章节标题样式 --------------------------
% 部分(Part)标题:新页+居中+特大号加粗
\titleformat{\part}{\newpage\centering\Huge\bfseries}{}{0em}{}
\titlespacing*{\part}{0pt}{0ex}{3.0ex} % 部分标题间距

% 章节(Chapter)标题:居中+大号加粗+带章节号
\titleformat{\chapter}{\centering\Large\bfseries}{\thechapter.}{1em}{}
\titlespacing*{\chapter}{0pt}{3.0ex}{2.0ex} % 章节标题间距

% 小节(Section)标题:大号加粗+带§符号
\titleformat{\section}{\centering\large\bfseries}{§\arabic{section}}{0.8em}{}
\titlespacing*{\section}{0pt}{2.5ex}{1.5ex} % 小节标题间距

% 自定义未编号章节(Explanation):封装chapter*,复用其样式
\newcommand{\explanation}[1]{\chapter*{#1}}



% -------------------------- 4. 段落与列表格式 --------------------------
\parindent=2em % 段落首行缩进2字符

% 有序/无序/描述列表统一格式:调整间距
\setenumerate[1]{itemsep=5pt, partopsep=0pt, parsep=\parskip, topsep=5pt}
\setitemize[1]{itemsep=5pt, partopsep=0pt, parsep=\parskip, topsep=5pt}
\setdescription{itemsep=5pt, partopsep=0pt, parsep=\parskip, topsep=5pt}

% 重新定义 \hl 命令
\definecolor{lightpink}{rgb}{1.0, 0.71, 0.76}
\newcommand{\hl}[1]{%
  \bgroup
  \markoverwith{\textcolor{lightpink}{\rule[-.5ex]{2pt}{2.5ex}}}%
  \ULon{#1}%
}


% -------------------------- 5. 页眉页脚配置 --------------------------
% -------------------------- 5. 页眉页脚配置 --------------------------
\setlength{\headheight}{15pt} % 解决 headheight too small 警告
\pagestyle{fancy} % 使用fancy样式
\fancyhf{} % 清空默认内容
\fancyhead[L]{\leftmark} % 左页眉:当前章节标题
\fancyhead[R]{\thepage} % 右页眉:页码
\renewcommand{\headrulewidth}{0.4pt} % 页眉下边框线宽
\renewcommand{\footrulewidth}{0pt} % 隐藏页脚下边框


% -------------------------- 6. 代码块样式(分语言定义) --------------------------
% 定义Mathematica代码样式(避免全局覆盖)
\lstdefinestyle{MathematicaStyle}{
    language=Mathematica, basicstyle=\tt, breaklines=true,
    keywordstyle=\bfseries\color{NavyBlue}, emphstyle=\bfseries\color{Rhodamine},
    commentstyle=\itshape\color{black!50!white}, stringstyle=\bfseries\color{PineGreen!90!black},
    columns=flexible, numbers=left, numberstyle=\footnotesize, frame=tb, breakatwhitespace=false
}

% 定义TeX代码样式(避免全局覆盖)
\lstdefinestyle{TeXStyle}{
    language=TeX, basicstyle=\ttfamily, breaklines=true,
    keywordstyle=\bfseries\color{NavyBlue}, emphstyle=\bfseries\color{Rhodamine},
    commentstyle=\itshape\color{black!50!white}, stringstyle=\bfseries\color{PineGreen!90!black},
    columns=flexible, numbers=left, numberstyle=\footnotesize, frame=tb, breakatwhitespace=false
}


% 可选:设置全局默认代码样式(按需启用)
% \lstset{style=MathematicaStyle}% 传进去后有 ../config/package.tex
    \begin{document}
\else% 如果定义了
\fi


\section{切线、速度与变化率}

初等几何课程已经告诉我们如何做圆的切线。但那做法依赖于圆的特殊几何性质,并没有提示做一般曲线的切线的方法。
初等几何着眼于具体地研究每一特殊图形的性质,而高等数学却致力于寻求普遍地解决问题的方法。
为此,首先引进坐标把几何问题“代数化”。

考察如下的典型问题。设$y = f(x)$是在$(a,b)$上有定义的函数,它表示$OXY$坐标系中的一段曲线。
我们希望过曲线$y = f(x) \ (x \in (a,b))$上的一点$P_0(x_0, f(x_0))$,做这曲线的切线(图0-3)。
为此,考虑曲线上的另一点$P(x, f(x))$。过这两点可以做一条直线——曲线的割线——$P_0P$,其斜率为:


\[
\frac{f(x) - f(x_0)}{x - x_0}
\]


\begin{figure}[h]
    \centering
    % The \graphicspath in config.tex handles the folder location
    \includegraphics[width=0.6\textwidth]{0_3.jpg}
    \caption{作曲线的割线}
    \label{fig:0_3}
\end{figure}


当点$P$沿着曲线变动时,割线$P_0P$的方位也随着变动;当$P$无限接近于$P_0$时,割线$P_0P$的极限位置就应该是曲线过$P_0$点的切线。在以后的课程中,我们将看到,对于相当普遍的函数(包括我们在中学学过的所有的初等函数),当$P$趋于$P_0$时,割线$P_0P$确实有一个极限位置。这就是说,可以做曲线过$P_0$点的切线,其斜率为:


\[
f'(x_0) = \lim_{x \to x_0} \frac{f(x) - f(x_0)}{x - x_0}
\]

我们把差商$\frac{f(x) - f(x_0)}{x - x_0}$的极限$f'(x_0)$称为导数或微商。


我们再来看一个属于运动学的问题。设物体沿$OX$轴运动,其位置$x$是时间$t$的函数

\[
x = f(t)
\]

如果运动比较均匀,那么我们可以用平均速度反映其快慢。在$[t_1, t_2]$这一段时间里的平均速度定义为:


\[
v_{[t_1, t_2]} = \frac{f(t_2) - f(t_1)}{t_2 - t_1}
\]

如果物体的运动很不均匀,那么平均速度就不能很好地反映物体运动的状况,
必须代之以在每一时刻$t_0$的瞬时速度$v(t_0)$。为了计算瞬时速度,
我们取越来越短的时间间隔$[t_0, t]$,以平均速度$\bar{v}_{[t_0, t]}$作为瞬时速度$v(t_0)$的近似值。
让$t$趋于$t_0$,平均速度$\bar{v}_{[t_0, t]}$的极限即为物体在时刻$t_0$的瞬时速度:


\[
v(t_0) = \lim_{t \to t_0} \frac{f(t) - f(t_0)}{t - t_0} = f'(t_0)
\]

与切线问题一样,我们又遇到了差商$\frac{f(t) - f(t_0)}{t - t_0}$的极限——导数(或微商)$f'(t_0)$。



速度问题只是更一般的变化率问题的一个例子。假设有一个随时间变化的量$x = f(t)$。我们把差商:


\[
\frac{f(t_2) - f(t_1)}{t_2 - t_1}
\]
称为这个量从时刻$t_1$到时刻$t_2$的平均变化率。
当量$x = f(t)$变化比较均匀时,平均变化率反映了它变化的快慢。
如果量$x = f(t)$的变化很不均匀,就需要用瞬时变化率来描述这个量的各个不同时刻的变化状况。
取接近时刻$t_0$的一小段时间,考察这段时间内的平均变化率。当$t$趋于$t_0$时平均变化率的极限就是量$x = f(t)$在时刻$t_0$的瞬时变化率:


\[
f'(t_0) = \lim_{t \to t_0} \frac{f(t) - f(t_0)}{t - t_0}
\]


\textbf{例1} 
设从时刻0到时刻$t$通过导线截面的电量是$q = f(t)$。电量的平均变化率就是平均电流强度:


\[
\bar{I}_{[t_1, t_2]} = \frac{f(t_2) - f(t_1)}{t_2 - t_1}
\]
而电量的瞬时变化率则表示在时刻$t_0$的瞬时电流强度:


\[
I(t_0) = f'(t_0) = \lim_{t \to t_0} \frac{f(t) - f(t_0)}{t - t_0}
\]


\textbf{例2} 
设容器内有某种放射性元素,其质量$m$随着时间$t$而变化:$m = f(t)$。因为放射性元素衰变的时候质量不断减少,所以质量的平均变化率总是负数:


\[
\frac{f(t_2) - f(t_1)}{t_2 - t_1} < 0
\]
平均变化率的绝对值被称为平均衰变速度。质量的瞬时变化率也是负数:


\[
f'(t_0) = \lim_{t \to t_0} \frac{f(t) - f(t_0)}{t - t_0} < 0
\]
瞬时变化率的绝对值被称为瞬时衰变速度。



上面考察的几个问题,涉及几何学、力学、电学和物质放射性,而在这些问题中都出现了差商的极限
————导数。


\[
f'(x_0) = \lim_{x \to x_0} \frac{f(x) - f(x_0)}{x - x_0}
\]

由此看来,对这样的极限进行研究很有必要。关于导数的计算,已经发展了一套行之有效的方法——微分法。
这将是我们进一步学习的重要内容。


\ifx\allfiles\undefined %若没定义主文件
\end{document}% 就打印这个结尾
\fi